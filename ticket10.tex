\section{Ticket 10: Тьюринг}
\myworries{Нет в билетах, но на лекциях было}
\label{sec-12}
\subsection{Арифметические отношения, их выразимость}
\label{sec-12-1}
\item Арифметическое отношение $R \in N_0^n$ выразимо в ФА, если $\exists a$ с $n$ свободными переменными $a(x_1, \dotsc, x_n)$, такая что:
\begin{enumerate}
\item Eсли $R(k_1\dots k_n)$, то $\vdash a(\overline{k_1}\dots \overline{k_n})$
\item Eсли $\lnot R(k_1..k_n)$, то $\vdash \lnot a(\overline{k_1} \dots \overline{k_n})$
\end{enumerate}
\subsection{Гёделева нумерация}
\label{sec-12-2}
\begin{center}
\begin{tabular}{lrl}
$a$ & $\Godel{a}$ & описание\\
\hline
$($        & $3$ & \\
$)$        & $5$ & \\
$,$        & $7$ & \\
$\lnot$    & $9$ & \\
$\to$      & $11$ & \\
$\lor$     & $13$ & \\
$\&$       & $15$ & \\
$\forall$  & $17$ & \\
$\exists$  & $19$ & \\
$x_k$      & $21 + 6 \cdot k$ & переменные\\
$f^n_k$    & $23 + 6 \cdot 2^k \cdot 3^n$ & n-местные функцион. символы (', +, *)\\
$P^n_k$    & $25 + 6 \cdot 2^k \cdot 3^n$ & n-местные предикаты (=)\\
\hline
\end{tabular}
\end{center}
Последовательность значков будем составлять так:\\
$a_1, \dotsc, a_n$ - наши простые числа, соответствующие символам, тогда $p_1^{a_1} * p_2^{a_2}, \dotsc, p_n^{a_n}$ - геделев нумерал строки, составленной из символов.

Если $a$ -- выражение, то $\Godel{a}$ - выражение в Геделевой форме. Тогда если $a$ -- выражение, $\overline{\Godel{a}}$ - это элемент предметного множества ФА, соответствующий нолику с количеством черточек, равным $\Godel{a}$.

Доказательство -- это последовательность простых чисел, возведенная
в геделевы нумералы выражений, являющихся составляющими док-ва, по
порядку. Аналогично с составлением строки из символов.

Тогда определим следующие операции с нумералами:
\begin{itemize}
\item $\plog(a, b) = \max n : a \perc b^n = 0$\\
Иногда вместо $b$ стоит $P_{b}$, где $P_{b}$ -- простое число с индексом $b$.\\
Функция берет геделев нумерал и достает у него $i$-й элемент последовательности
\item $len = \max n : a \perc p_n$\\
Возвращает длину строки доказательства
\item $s@t = p_1^{\plog(s, 1)} * \dots  * p_{len(s)}^{\plog(s, len(s))} *
p_{len(s+1)}^{\plog(t, 1)} * \dots  * p_{len(s)_len(t)}^{\plog(t, len(t))}$\\
Конкатенация строк
\end{itemize}
\subsection{Машина Тьюринга}
\label{sec-12-3}
Машина тьюринга состоит из ленты, головки, регистра состояния и конечной таблицы состояний.\\
Более формально, это 7-кортеж: $<Q, \Gamma, b, \sum, \sigma, q_0, F>$\\
Конечный список состояний, конечный алфавит, пустой символ из алфавита, символы, которые мы можем писать (из $\Gamma \backslash b$), функция таблицы состояний, начальное состояние из $Q$, конечное состояние из $Q$.
\begin{itemize}
\item Лента -- бесконечный двусвязный список, в каждой ячейке которого содержится символ из конечного алфавита, в котором также есть пустой символ (тут и далее), которым изначально заполнена вся лента
\item Головка может находиться над элементом, писать в него и читать из него символ. Может двигаться влево-вправо (или двигать ленту, неважно)
\item Регистр состояния хранит состояние -- элемент из конечного множества состояний машины. Есть особые состояния - стартовое и конечные.
\item Таблица состяний -- таблица, хранящая данные о функции смены состояния -- $foo: \Gamma \times Q \to \Gamma \times Q \times \{left, this, right\}$.\\
Функция берет текущее состояние, читает символ на головке, потом получает тройку, пишет новый символ, перемещается по третьему элементу, выставляет новое состояние. Если состояние конечное, то она останавливается.
\end{itemize}
Мы будем придерживаться нотации $<\_, \_,  , \_, \_, S, F>$.
\myworries{Что это за хуйня?}
\subsection{Проблема останова}
\label{sec-12-4}
Дано описание процедуры и входные данные. Функция $P(a, b)$ определяет, остановится ли $a$ на входных данных $b$. Существует ли $P$?
\begin{itemize}
\item Проблема останова неразрешима на машине Тьюринга:\\
Пусть $P$ существует.\\
Тогда $S(x) = P(x, x)$ остановится ли функция на своем же коде\\
$MyProg(x) = if S(x) then while(true)\{\} else 1$ \myworries{Пиздос тут код какой-то, надо бы нахехлатехать}
Рассмотрим $MyProg(`MyProg)$:\\
Если оно остановится, то первое условие выполнено, тогда оно не остановится и наоборот.\\
Значит, $P$ не существует.
\end{itemize}
\subsection{Выводимость и рек. функции - Тьюринг}
\label{sec-12-5}
\subsubsection{Выражение машин Тьюринга через рекурсивные функции}
\label{sec-12-5-1}
Мы хотим доказать, что если у нас есть какая-нибудь процедура, которую можно выразить в Тьюринге, то мы можем ее сделать и в формальной арифметике (рекурсивные функции представимы).\\
Введем обозначение $\langle st,tape,pos \rangle = 2^{st}*3^{tape}*5^{pos}$\\
Такая тройка -- основная характеристика машины в данный момент.\\
Будем называеть ее текущим полным состоянием, например.\\
$st$, $tape$, $pos$ -- геделевы нумералы, $st$ -- нумерал из $1$ элемента с состоянием, $tape$ - строка, обозначающая ленту (бесконечные слева и справа не входят), $pos$ - позиция в ленте.
\begin{itemize}
\item $p: \langle st, a \rangle \to \langle st, a, dir \rangle$\\
Принимает $\langle st, a \rangle$, декодит, лезет в $\sigma$ машины тьюринга, достает новые значения, делает из них $\langle ,,\rangle$ (\myworries{Шта}), отдает.
\item $t: \langle st \rangle \to 0 | 1$\\
Определяет, терминально ли наше состояние ($0$ если терминально)
\item $\epsilon$  -- пустой символ (у нас  )
\item $pb$, $pc$ кодируют $\beta$-функцией последовательность инпутов в последовательность аутпутов. $\beta(p_{b}, p_{c}, x) = p(x)$
\item $tb$, $tc$ аналогично кодируют $t$
\item $R\template{f, g}(\langle s_{st}, s_{tape}, s_{pos} \rangle, \epsilon , pb, pc, tb, tc, y)$
Запускает машину Тьюринга от стартового состояния, заранее говоря ей, сколько шагов (y) она должна сделать.\\
Возвращает тройку $\langle st, tape, pos \rangle$
\item Определим $f$, $g$
\begin{enumerate}
\item Дополнительные функции
\begin{itemize}
\item $os(prev) = \plog(prev, 1)$\\
Текущее состояние
\item $ot(prev) = \plog(prev, 2)$\\
Лента
\item $op(prev) = \plog(prev, 3)$\\
Позиция головки в ленте
\item $nextstate(pb, pc, prev) = \beta(pb, pc, 2^{os(prev)} * 3^{\plog(ot(prev), op(prev))}$\\
Реализует функцию p
\item $st(pb, pc, prev) = \plog(nextstate(pb, pc, prev), 1)$\\
Новое состояние.
\item $sym(pb, pc, prev) = \plog(nextstate(pb, pc, prev), 2)$\\
Символ который нужно писать
\item $dir(pb, pc, prev) = \plog(nextstate(pb, pc, prev), 3)$\\
Направление для перехода головки
\item $repl(pb, pc, prev) = (ot(prev) / (P_{op})^{\plog(ot(..), op(..))}) * (P_{op})^{sym}(..)$
Возвращает ленту, в которой удален символ в позиции op, и добавлен новый символ в эту же позицию.
\end{itemize}
\item $f$ -- возвращает полное состояние машины\\
$f(\langle start_{state} \rangle, \epsilon , pb, pc, tb, tc) = \langle start_{state} \rangle$
\item $g$ -- возвращает новое полное состояние из машины после перехода (пометка: $0$ - $nothing$, $1$ - $right$, $2$ - $left$ все фукнции вызываются с аргументом $prev, \langle start_{state} \rangle$ не используется)\\
$g(\langle start_{state} \rangle, \epsilon , pb, pc, tb, tc, y, prev) =$
\begin{center}
\begin{tabular}{lrl}
Condition & Result & Descr\\
\hline
$dir = 0$        & $\langle st, repl, op \rangle$ & nothing\\
$dir = 1 \& len(repl) = op$ & $\langle st, repl @ 2^{\epsilon}, op + 1 \rangle$ & tape end\\
$dir = 1$        & $\langle st, repl, op + 1 \rangle$ & move right\\
$dir = 2 \& op = 0$    & $\langle st, 2^{\epsilon} @ repl, op - 1$ & tape start\\
$dir = 2$      & $\langle st, repl, op - 1 \rangle$ & move left\\
\hline
\end{tabular}
\end{center}
\end{enumerate}
\item $steps$ -- функция, определяющая необходимое кол-во шагов

$steps(\langle \const{start\_state} \rangle, \epsilon , pb, pc, tb, tc) =$\\
$\mu \langle \beta(tb, tc, \plog(R\template{f, g}, 1))\rangle (\langle \const{start\_state} \rangle, \epsilon , pb, pc, tb, tc)$\\
Она найдет такое минимальное $k$, что состояние $\plog(R\template{f, g}(args, k), 1)$ терминально.
\end{itemize}
\subsubsection{Выражение программы по проверке доказательства в машине тьюринга}
\label{sec-12-5-2}
\begin{itemize}
\item $Emulate(input, prog) = \plog(R\template{f, g}(\langle \Godel{S}, input, 0 \rangle,  , pb, pc, tb, tc, steps(-//-)), 1) == F$\\
Функция проверяет, правда ли получившееся терминальное состояние - ок.\\
Можем давать программу такую, что она заканчивается в терминальном $F(inish)$ или в терминальном $FAIL$\\
Дает в качестве аргумента функцию перехода, $pb$, $pc$ выражают $prog$
\item
$\begin{multlined}[t]
Proof(term, proof) = Emulate(proof, MY\_PROOFCHECKER) \\
\&\& (\plog(proof, len(proof)) = term)
\end{multlined}$\\
Проверяет, что доказательство $p$ заканчивается корректно и его последний элемент -- то, что мы доказываем.
\item Любая представимая в ФА функция является рекурсивной\\
Пусть $f$ представима\\
Пусть $f(x_1, \dotsc, x_n) = b$, тогда $\vdash \phi(\overline{x_1}, \dotsc \overline{x_n}, \overline{b})$\\
Всегда можно построить рекурсивную функцию $G_{\phi(x_1, \dotsc, x_n, b, p)}$, утверждающую, что $p$ -- гёделев номер вывода предиката $\phi(\overline{x_1}, \dotsc, \overline{x_n}, \overline{b})$\\
Мы делаем это обычным перебором чисел, проверяем вывод нашей программой из домашнего задания, выраженную в тьюринге, а потом в
рекурсивных функциях.\\
Тогда $f$ в рекурсивных функциях выражается так:
$f(x_1\dots x_n) = \plog(\mu\template{S\template{G_{\phi}, U_{n+1}^1, \dotsc, U_{n+1}^n, $\\
$\plog(U_{n+1}^{n+1}, 1),$\\
$\plog(U_{n+1}^{n+1}, 2)}}(x_1, \dotsc, x_n), 1)$\\
Такая функция берет $\plog(1)$ от первого такого минимального геделева номера $k$ (геделева пара из двух элементов - $\langle b, p \rangle$)), что:
\begin{gather*}
S\template{G_{\phi}, U \dots}(x_1, \dotsc, x_n, k) = 0
\shortintertext{Это значит, что:}
G_{\phi}(x_1, \dotsc, x_n, \plog(k, 1), \plog(k, 2)) = 0 \\
\shortintertext{И значит, что:}
G_{\phi}(x_1, \dotsc, x_n, b, p) = 0
\shortintertext{То есть $p$ -- вывод:}
\phi(\overline{x_1}, \dotsc, \overline{x_n}, \overline{b}). \\
\shortintertext{Этот гёделев номер -- $b$}
\end{gather*}
\end{itemize}