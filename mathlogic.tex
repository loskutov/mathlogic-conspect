% Created 2015-01-21 Ср 21:06
\documentclass[a4paper, fleqn, draft]{article}
\usepackage[english,russian]{babel}
\usepackage{euler}
\usepackage{fontspec}
\usepackage{amssymb}
\usepackage{amsmath}
\usepackage{amsthm}
\usepackage[left=2cm,right=1.6cm,top=2cm,bottom=2cm,bindingoffset=0cm]{geometry}
\usepackage{xcolor}
\usepackage{epigraph}
\usepackage{mathtools}
\usepackage{nicefrac} % ½ и его друзья
\usepackage{proof} % M. P. и его друзья
\usepackage[linkbordercolor={0 0 1}, pdftitle={Курс матлогики по Штукенбергу Д. Г.}]{hyperref}
\usepackage{comment}
\usepackage{stmaryrd} % ⟦⟧
\usepackage{clrscode3e}
\usepackage{enumitem} % всякие модные enumerate-ы (latin letters, roman numbers etc)
\usepackage{pict2e} % templates
%\usepackage{microtype} % модная микротипографика, нифига не работающая с XeLaTeX (ну, почти)
%\setmainfont[Ligatures=TeX,SmallCapsFont={Times New Roman}]{Palatino Linotype}
%\setmainfont[Ligatures=TeX]{Times New Roman}
%\usepackage{polyglossia}
%\setdefaultlanguage[babelshorthands=true]{russian}
\defaultfontfeatures{Ligatures={TeX},Renderer=Basic}
\setmainfont[Ligatures={TeX,Historic}]{DejaVu Sans}
\setmonofont{DejaVu Sans Mono}
%\hyphenpenalty=5000


\renewcommand{\le}{\leqslant} % ⩽
\renewcommand{\leq}{\leqslant} % ⩽
\renewcommand{\ge}{\geqslant} % ⩾
\renewcommand{\geq}{\geqslant} % ⩾
\renewcommand{\phi}{\varphi}
\renewcommand{\epsilon}{\varepsilon}

\newcommand{\true}{\texttt{И}}
\newcommand{\false}{\texttt{Л}}
\newcommand{\R}{\mathbb{R}}
\newcommand{\Z}{\mathbb{Z}}
\newcommand{\N}{\mathbb{N}}

% Declaring some new operators
\DeclareMathOperator{\plog}{plog}
\DeclareMathOperator{\Int}{Int}
\DeclareMathOperator{\Proof}{Proof}
\renewcommand{\land}{\mathbin{\&}}
\newcommand{\perc}{\mathbin{\%}}

\newcommand{\ltemplate}{\begin{picture}(5,7)
\put(.5,4){\line(1,-1){5}}
\put(.2,4.1){\line(1,-1){5.2}} % just to make it more bold lol
\put(.5,4){\line(1,1){5}}
\put(.2,3.9){\line(1,1){5.2}} % just to make it more bold lol
\end{picture}}
\newcommand{\rtemplate}{\begin{picture}(5,7)
\put(4.5,4){\line(-1,-1){5}}
\put(4.8,4.1){\line(-1,-1){5.2}} % just to make it more bold lol
\put(4.5,4){\line(-1,1){5}}
\put(4.8,3.9){\line(-1,1){5.2}} % just to make it more bold lol
\end{picture}}
\newcommand{\template}[1]{\ltemplate #1\rtemplate}

\newcommand{\starsection}[1]{\addtocounter{section}{1}\setcounter{subsection}{0}%
    \addcontentsline{toc}{section}{\hspace{-2em}\texorpdfstring{$\star\quad$\thesection\hspace{1em}}{}#1}%
{\vspace{1.5em}\par\noindent\bfseries\Large \thesection$^\star$\hspace{0.5em} #1}\par\vspace{1em}\noindent}

% some magic for Godel numerals
\newbox\gnBoxA
\newdimen\gnCornerHgt
\setbox\gnBoxA=\hbox{$\ulcorner$}
\global\gnCornerHgt=\ht\gnBoxA
\newdimen\gnArgHgt
\def\Godel #1{%
\setbox\gnBoxA=\hbox{$#1$}%
\gnArgHgt=\ht\gnBoxA%
\ifnum     \gnArgHgt<\gnCornerHgt \gnArgHgt=0pt%
\else \advance \gnArgHgt by -\gnCornerHgt%
\fi \raise\gnArgHgt\hbox{$\ulcorner$} \box\gnBoxA %
\raise\gnArgHgt\hbox{$\urcorner$}}

\newcommand{\defeq}{\coloneqq}
\newcommand{\s}[1]{\texttt{#1}}
\newcommand{\xl}{$\lambda$}
\newcommand{\+}{\lambda}
\newcommand{\bredmath}{\ \longrightarrow_\beta\ }
\newcommand{\bred}{$\bredmath$}
\newcommand{\mbred}{$\ \longrightarrow\!\!\!\!\rightarrow_\beta\ $}
\newcommand{\lid}[1]{\textit{#1}}
\newcommand{\concat}{\hat{\ \ }}

\newcommand\myworries[1]{\textcolor{red}{#1}}

\def\ra{\rightarrow}

\tolerance 1000 % чтобы не очковал переносить

\renewcommand{\theenumii}{\asbuk{enumii}}
\AddEnumerateCounter{\asbuk}{\@asbuk}{ы}

\DeclareRobustCommand{\divby}{%
  \mathrel{\vbox{\baselineskip.65ex\lineskiplimit0pt\hbox{.}\hbox{.}\hbox{.}}}%
}

\author{Daniyar Itegulov, Aleksei Latyshev, Ignat Loskutov}
\date{\today}
\title{Курс математической логики по Штукенбергу Д.~Г.}

\begin{document}
\theoremstyle{definition}
\newtheorem*{definition}{Определение}%[section]
\newtheorem*{example}{Пример}
%\theoremstyle{theorem}
\newtheorem{theorem}{Теорема}[section]
\newtheorem{axiom}{Аксиома}[section]
\newtheorem{lemma}[theorem]{Лемма}

\maketitle
\setcounter{tocdepth}{2}
\tableofcontents

%Должно отображаться корректно: $x_1$,x₂x₃x_n, θ, \exists , ∑, ∉
\epigraph{%
    лан, всё не книжку верстаем))))000}
  {некто Игнат Лоскутов о качестве вёрстки}

Mykhail Volkhov, 2538, 2014Sep-2015Jan\\
Я не отвечаю за верность написанного - много информации
я придумал сам, много достал из недостоверных источников.
\include{base}
\include{defenitions}
\section{Исчисление высказываний}
\label{sec-3}
\subsection{Определения (исчисление, высказывание, оценкa\ldots{})}
\label{sec-3-1}
Формальная система с алгеброй Яськовского $J_{0}$ в качестве модели (множество
истинностных значений $\lbrace 0, 1 \rbrace$). Формальная теория нулевого порядка, кванторов
нет, предикаты - это пропозициональные переменные.
\subsection{Общезначимость, доказуемость, выводимость}
\label{sec-3-2}
\begin{itemize}
\item Формула называется общезначимой в теории с моделью если эта формула
верна в любой модели данной теории. Например в ИВ формула называется
общезначимой если любая оценка формулы на любых значениях пропозициональных
переменных (что является моделью ИВ) возвращает истину. В ИИВ же общезначимостью
называется существование формулы во всех мирах всех возможных моделей крипке
(которые являются моделью для ИИВ).
\item Доказуемость - свойство формулы в теории, значащее, что существует
доказательство для этой формулы. Доказательство для теории тоже определяется
по разному (последовательность утверждений, каждое из которых есть аксиома
или следует по правилу вывода из предыдущих в ИВ, дерево с выводами в $S\infty$)
\end{itemize}
\subsection{Схемы аксиом и правило вывода Modus Ponens}
\label{sec-3-3}
Схемы аксиом:
\begin{enumerate}
\item $\alpha \to \beta \to \alpha$
\item $(\alpha \to \beta) \to (\alpha \to \beta \to \gamma) \to (\alpha \to \gamma)$
\item $\alpha \to \beta \to \alpha \land \beta$
\item $\alpha \land \beta \to \alpha$
\item $\alpha \land \beta \to \beta$
\item $\alpha \to \alpha \lor \beta$
\item $\beta \to \alpha \lor \beta$
\item $(\alpha \to \beta) \to (\gamma \to \beta) \to (\alpha \lor \gamma \to \beta)$
\item $(\alpha \to \beta) \to (\alpha \to \lnot \beta) \to \lnot \alpha$
\item $\lnot \lnot \alpha \to \alpha$
\end{enumerate}

Единственное правило вывода Modus Ponens (M.P.):
\[\infer{\beta}{\alpha & (\alpha \rightarrow \beta)}\]
\subsection{Теорема о дедукции}
\label{sec-3-4}
\begin{theorem}
	$\Gamma, \alpha \vdash \beta \Leftrightarrow \Gamma \vdash \alpha \to \beta$
\end{theorem}
\begin{proof}
\leavevmode
$\Rightarrow$)
Нужно переместить последнее предположение вправо.
Будем доказывать индукцией по доказательству. Обработаем три случая в переходе:
\begin{enumerate}
\item Аксиома или предположение \\
$A$ \\
$A\to \alpha \to A$ \\
$\alpha \to A$
\item Modus Ponens \\
По предположению индукцию уже было доказано $\alpha \to A$, $\alpha \to A \to B$ \\
$(\alpha \to A)\to (\alpha \to A \to B)\to (\alpha \to B)$ \\
$(\alpha \to A\to B)\to (a\to B)$ \\
$\alpha \to B$
\item Само это выражение \\
$\alpha\to \alpha$ умеем доказывать
\end{enumerate}
$\Leftarrow$) Если нужно переместить влево, то перемещаем и добавляем следующее: \\
$A\to B$ (последнее) \\
$A$    (перемещенное) \\
$B$
\end{proof}

\subsection{Корректность исчисления высказываний относительно алгебры Яськовского}
\label{sec-3-5}
\begin{itemize}
\item Индукцией по доказательству -- если аксиома, то она
тавтология, все ок. Если модус поненс, то таблица
истинности для импликации и все ок
\end{itemize}

%%% Local Variables:
%%% mode: latex
%%% TeX-master: "mathlogic"
%%% TeX-engine: xetex
%%% End:
\include{ticket2}
\include{ticket3}
\include{ticket4}
\include{ticket5}
\include{ticket6}
\include{ticket7}
\include{ticket8}
\include{ticket9}
\include{ticket10}
\include{ticket11}
\include{ticket12}
%\include{ticket13}
\include{ticket14}
%\include{ticket15}
%\include{ticket16}
\end{document}

%%% Local Variables:
%%% config: utf-8
%%% mode: latex
%%% TeX-engine: xetex
%%% End: