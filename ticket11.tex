\section{Ticket 11: 1т о неполноте}
\label{sec-13}
\subsection{Непротиворечивость, ω-непротиворечивость}
\label{sec-13-1}
\begin{itemize}
\item Теория непротиворечива, если в ней нельзя вывести
одновременно a и \lnot a (что аналогично невозможности
вывести a\&\lnot a).
\item Теория ω-непротиворечива, если из %\forall \phi(x) \vdash \phi(x\textasciitilde{})$ следует
⊬ \exists p\lnot \phi(p). Проще говоря, если мы взяли
формулу, то невозможно вывести одновременно \exists x\lnot A(x)
и A(0), A(1), \dots
\item Лемма о w-\# и обычной непротиворечивости
Если теория w-непротиворечива, то она непротиворечива
\phi = x=x \to x=x
Такая формула очевидно доказуема (A \to A)
$\vdash \phi[x:=k] k \in N₀$
Но недоказуемо \exists x\lnot (x=x\to x=x)
А в противоречивой теории доказуемо все
\end{itemize}
\subsection{Прервая теорема о неполноте}
\label{sec-13-2}
Определим отношение W_1(x, p), истинное тогда и только тогда,
когда x - геделев номер формулы \phi с единственным свободным
аргументом x, а p - геделев номер доказательства \phi("\phi"). Это
отношение выразимо в ФA, потому что мы просто пихаем это в наш
Proof, а его мы выразили через рекурсивные функции, а они
представимы.
Пусть его выражает w_1(x, p);
Рассмотрим формулу σ = \forall p\lnot w_1(x, p) - для любого доказательства
оно не является доказательством самоприменения \phi, то есть
самоприменение \phi недоказуемо.
То есть если σ(`a\textasciitilde{}) истинно, то a(`a\textasciitilde{}) недоказуемо.
В нашем случае если σ(`a\textasciitilde{}) истинно, то σ(`σ\textasciitilde{}) недоказуемо.
\begin{enumerate}
\item Если формальная арифметика непротиворечива, то недоказуемо σ(`σ\textasciitilde{})
\begin{enumerate}
\item Пусть $\vdash σ(`σ\textasciitilde{})$, тогда найдется геделев номер ее док-ва p,
тогда W_1(`σ, p), то есть $\vdash w_1(`σ\textasciitilde{}, p\textasciitilde{})$.
\item С другой стороны,
$\vdash σ(`σ\textasciitilde{})$
$\vdash \forall p\lnot w_1('σ\textasciitilde{}, p)$
\forall p\lnot w_1(`σ\textasciitilde{}, p) \to \lnot w_1(`σ\textasciitilde{}, p\textasciitilde{})
\lnot w_1(`σ\textasciitilde{}, p\textasciitilde{})
Тогда ФА противоречива.
\end{enumerate}
\item Если формальная арифметика w-непротиворечива, то недоказуемо \lnot σ(`σ\textasciitilde{})
Пусть $\vdash \lnot σ(`σ\textasciitilde{})$, то есть $\vdash \lnot \forall p\lnot w_1(`σ\textasciitilde{}, p)$, что значит \exists p.w_1(`σ\textasciitilde{}, p)
Найдется такой q, что $\vdash w_1(`σ\textasciitilde{}, q\textasciitilde{})$, потому что если бы не нашелся,
это бы значило доказуемость для каждого q \lnot w_1(`σ\textasciitilde{}, q\textasciitilde{}), тогда по
ω-непротиворечивости было бы не доказуемо \exists p\lnot \lnot w_1(`σ\textasciitilde{}, p)
То q, что мы нашли - это номер доказательства  σ(`σ\textasciitilde{}), что и
утверждает выражение $\vdash w_1(`σ\textasciitilde{}, q\textasciitilde{})$. Но мы предполагали, что $\vdash \lnot σ(`σ\textasciitilde{})$.
Противоречие.
\end{enumerate}

Нормальное доказательство общезначимости:
Я не знаю, зачем нам второй пункт, но из первого следует, что если
наша теория w-непротиворечива, то она непротиворечива (по лемме выше),
значит в ней недоказуемо σ(`σ\textasciitilde{}), то есть \forall p\lnot w_1(`σ\textasciitilde{}, p), то есть
по корректности последнее выражение И, но это и есть в точности определение
σ(`σ\textasciitilde{}).

Ненормальное д-во общезначимости:
Итого мы доказали, что если формальная арифметика ω-непротиворечива,
то в ней не доказуемо ни σ(`σ\textasciitilde{}) ни \lnot σ(`σ\textasciitilde{}). Одно из них точно тавтология
(в формуле нет свободных переменных). Тогда ФА неполна при условии
ω-непротиворечивости.

Другое доказательство общезначимости:
\lnot σ(`σ\textasciitilde{}) недоказуема
[σ(`σ\textasciitilde{})] = [\forall p\lnot w_1(`σ\textasciitilde{}, p)] =
\begin{enumerate}
\item И если [\lnot w_1(`σ\textasciitilde{}, a)] = И для какого-то а
\item Л иначе
\end{enumerate}

Это значит, что
И если [w_1(`σ\textasciitilde{}, a)] = Л
[w_1(`σ\textasciitilde{}, a)] = Л, докажем от противного
Пусть [σ(`σ\textasciitilde{})] = Л,
[\forall p\lnot w_1(`σ\textasciitilde{}, p)] = Л
[\lnot \forall p\lnot w_1(`σ\textasciitilde{}, p)] = И
[\exists p.w_1(`σ\textasciitilde{}, p)] = И
[w_1(`σ\textasciitilde{}, a)] = И для какого-то а
то есть a доказывает σ(`σ\textasciitilde{})
???

тогда по определению w_1 существует
доказательство σ(`σ\textasciitilde{}),
\subsection{Пример w-противоречивой, но непротиворечивой теории (при усл. непрот. ФА)}
\label{sec-13-3}
Добавим в ФА аксиому Г: \lnot σ(`σ\textasciitilde{})
Тогда по контрпозиции 1п2 она w-противоречива.
Если бы мы могли доказать противоречивость нашей системы, то
ФА была бы противоречива, тогда хз
$\lnot σ(`σ\textasciitilde{}) \vdash σ(`σ\textasciitilde{})\&\lnot σ(`σ\textasciitilde{})$
$\vdash σ(`σ\textasciitilde{})$
Но мы предположили что \lnot σ(`σ\textasciitilde{})
\subsection{Форма Россера}
\label{sec-13-4}
Если формальная арифметика непротиворечива, то в ней найдется
такая формула \phi, что ⊬\phi и ⊬\lnot \phi

