\section{Ticket 16: неполнота ФА}
\label{sec-18}
\subsection{Теорема о трансфинитной индукции}
\label{sec-18-1}
Пусть есть формула с одной свободной переменной a(x)
a истинна, если
\begin{enumerate}
\item a(0)
\item Если для любого конечного p - ординала мы можем
показать следование \{ q < p => a(p) \}, то a(p) истинно.
\end{enumerate}

Без док-ва, не требуется.

\subsection{Построение S∞}
\label{sec-18-2}
Мы строим еще одну теорию I порядка.
По сути, мы вкладываем ФА в нашу теорию так, что
любое доказательство ФА работает в S∞ и мы можем доказать
доказать непротиворечивость любого "импортированного" д-ва

\begin{enumerate}
\item Формулы:
Оставим связки \forall x, \lor , \lnot
Заметим, что \{\lor , \lnot \} полно для \{0, 1\}.
\item Доказательство
Доказательством является дерево утверждений, в
узлах которого правли, причем если дерево
растет вверх, то правила действуют сверху вниз.
\item Аксиомы:
\begin{enumerate}
\item все термы ФА без переменных типа θ_1=θ₂ (корректные)
\item все термы вида \lnot (θ_1=θ₂) если [θ_1]\ne [θ₂] (некорректные, все остальн.)
\end{enumerate}
\item Правила:
Примечание: в правилах используются боковые формулы,
они могут отсутствовать. Это сделано для формализации
того факта, что мы можем применять правило для двух любых
элементов нашей дизъюнкции или вроде того.
Примечение: org-mode подчеркивает a, если "$_{\text{a}}$\_"
\begin{enumerate}
\item Структурные
\begin{enumerate}
\item Перестановка
\uline{a\lor b\lor γ\lor σ}
a\lor γ\lor b\lor σ
\item Сокращение
\uline{a\lor b\lor b\lor γ}
a\lor b\lor γ
\end{enumerate}
\item Сильные
\begin{enumerate}
\item Ослабление
\uline{γ}
a\lor γ
\item Де-Морган
\uline{(\lnot a)\lor γ$_{\text{(\lnot b)}}$\lor γ}
\lnot (a\lor b)\lor γ
\item Отрицание
\uline{a\lor γ\_\_}
(\lnot \lnot a)\lor γ
\item Квантификация
\uline{\lnot a(t)\lor γ}
\lnot \forall x.a(x)\lor γ
\item Бесконечная индукция
\uline{a(0\textasciitilde{})\lor γ$_{\text{a}}$(1\textasciitilde{})\lor γ} \dots  \uline{a(r\textasciitilde{})\lor γ}\dots
\forall xa(x)\lor γ
\end{enumerate}
\item Сечение (для облегчения жизни)
\uline{γ\lor a$_{\text{\lnot a\lor δ}}$}
γ\lor δ
\end{enumerate}
\item Порядки
Каждой формуле в дереве соответствует порядок, причем
посылке и заключению (выше и ниже \uline{\_}) слабого правила
вывода соответствует один порядок, а порядковое число,
отнесенное заключению сильного правила или сечения, больше
порядковых чисел, отнесенных соотвтетствующим посылкам.
Порядковые числа - это ординалы, они могут быть достижимыми,
но не конечными - пусть формула какая-нибудь околорекурсивная
типа Ф(x): Ф(0) = A \to A, Ф(1) = A \to A \to A, Ф(2) = A \to A \to A \to A.
Тогда пусть мы хотим доказать \forall x.Ф(x) - по бесконечной
индукции порядок термов будет увеличиваться, а порядок
\forall x.Ф(x) будет w. Именно из-за этого факта мы используем
в доказательстве теоремы об устранении сечений трансфинитную
индукцию по порядку - ведь обычной индукции мало для порядков
больших w.
\item Степень
Степень сечения - количество связок в \lnot a.
Степень доказательства - наибольшая степень сечения в дереве.
Степень всегда конечна - любая формула в ФА содержит конечное
число связок, а при трансляции нет возможности увеличить
их количество. Тогда трансфинитная индукция по термам, в
которых в сечении количество связок растет до бесконечности,
невозможно.

\item Нитью называется последовательность формул от начальной до
конечной. Все нити в доказательстве конечны, поскольку если в начальной
формуле стоит ординал, последовательность в нити не возрастает, а эти
числа убывают с применением строгого правила или сечения. Мы знаем,
что строго убывающая бесконечная последовательность ординалов не существует.
Добавим правило, что последовательность применения слабых правил подряд
была всегда конечна
\item Теорема в S∞ - выражение, которое может стоять в заключительной формуле
вывода
\end{enumerate}
\subsection{Теоремы об эквивалентности ФА и S∞}
\label{sec-18-3}
\subsubsection{Лемма 1: В S∞ выводимо A\lor \lnot A}
\label{sec-18-3-1}
А либо корректна, либо некорректна, тогда
\uline{A}
\lnot A\lor A     ослабление

\uline{\lnot A}
\uline{A\lor \lnot A}   ослабление
\uline{\lnot A\lor A}   перестановка
\subsubsection{Лемма 2: В S∞ выводимо s\ne t\lor \lnot A(s)\lor A(t)}
\label{sec-18-3-2}
Если выводимо A(s), s = t, то выводимо A(t) (все вхождения меняем)
если s=t, то выводимо \lnot A(t)\lor A(t), потом сделаем ослабление
если s\ne t, то она аксиома (некорректная) и ослабим.
\subsubsection{Лемма 3: всякая выводимая в S замкнутая формула А является теоремой S∞}
\label{sec-18-3-3}
Докажем, что если что-то доказуемо в ФА, то его эквивалент
доказуем и в S∞.
$\vdash ₚₐA => \vdash ₛA'$
Схема док-ва
Рассмотрим доказательство в ФА, оно состоит из
\beta_1'\dots \beta_n', оттранслируем каждое в \beta_i \in S∞ (по полноте \{\lnot , \lor \} это возможно)
Тогда можно сделать дерево, в котором начальные формулы - аксиомы S,
а правила вывода - MP, GEN.

Рассмотрим формулу A = \beta_i
\begin{enumerate}
\item B \to C \to B, те \lnot B \lor (\lnot  C \lor B)
Из замкнутости A следует замкнутость B
\lnot B \lor B выводима по Л1, тогда ослабим с \lnot C, переставим.
\item (B \to C) \to (B \to C \to D) \to B \to D
\lnot (\lnot B \lor C) \lor (\lnot (\lnot B \lor (\lnot C \lor D)) \lor (\lnot B \lor D))
По Л1 выводимо \lnot (\lnot B \lor C) \lor (\lnot B \lor C),
(\lnot B \lor \lnot C \lor D) \lor \lnot (\lnot B \lor \lnot C \lor D)
Тогда можно по перестановке, сечению (с С) и сокращению
доказать (B \to C \to D) \to (B \to C) \to (B \to D)
что одно и то же, см дедукцию в предикатах
\item (B \to C) \to (B \to \lnot C) \to \lnot B
\lnot (\lnot B \lor C) \lor \lnot (\lnot B \lor \lnot C) \lor \lnot B
\begin{enumerate}
\item \lnot B \lor B                        л1
\item \lnot \lnot \lnot B \lor B                      отрицание
\item \lnot (\lnot \lnot B \lor C) \lor \lnot \lnot \lnot B \lor B         ослабление
\item \lnot \lnot \lnot B \lor \lnot (\lnot \lnot B \lor C) \lor B         перестановка
\item \lnot \lnot \lnot B \lor B \lor \lnot \lnot C                аналогично + еще перестановка
\item \lnot C \lor \lnot \lnot C                      лемма
\item \lnot C \lor B \lor \lnot \lnot C                  ослабление + перестановка
\item \lnot (\lnot \lnot B \lor C) \lor B \lor \lnot \lnot C          де-морган от 6 и 8
\item \lnot \lnot C \lor \lnot (\lnot \lnot B \lor C) \lor B          перестановка 8
\item \lnot (\lnot \lnot B \lor \lnot C) \lor \lnot (\lnot \lnot B \lor C) \lor B де-морган от 4 и 9
\end{enumerate}
Ну вот мы доказали что-то очень похожее на то, что нужно было.
Там контрпозиция, шмяк шмяк, готово.
\item Видимо, примерно все формулы так доказываются.
\item \forall x.B(x) \to B(t)
\lnot \forall x.B(x) \lor B(t)
По л1 \lnot B(t) \lor B(t), потом квантификация по 1 элем.
\item B(t) \to \exists x.B(x)
B(t) \to \lnot \forall x.B(x) (что заметно отличается от \forall x.\lnot B(x))
\lnot B(t) \lor \lnot \forall x.B(x)
Не, я не знаю. Но точно можно! По бесконечной индукции мож как-то.
Или там сечение хитрое.
\end{enumerate}

А еще есть аксиомы ФА
\begin{enumerate}
\item a = b \to a' = b'
\lnot (a = b) \lor a' = b'
\begin{itemize}
\item если a = b, то a' = b', это аксиома S∞,
тогда по ослаблению добавим \lnot (a = b)
\item если a \ne  b, то она же и аксиома
\end{itemize}
\item a = b \to a = c \to b = c
a \ne  b \lor a \ne  c \lor b = c
a \ne  b \lor \lnot (x = c) @ b \lor (x = c) @ c     по лемме 2
\item a' = b' \to a = b
Аналогично 1
\item \lnot (a' = 0)
Аксиома, поскольку a' всегда имеет с 0 разные значения
\item a + b' = (a + b)'
TODO
\item a + 0 = a
Аксиома, поскольку это вседа равенство
\item a * 0 = 0
аналогично 6
\item a * b' = a * b + a
TODO
\item \phi[x:=0] \& \forall x.(\phi \to \phi[x:=x']) \to \phi
\lnot B(0) \lor \lnot \forall x(\lnot B(x)\lor B(x')) \lor B(0)         лемма 1 и перестановка
\lnot B(0) \lor \lnot (\lnot B(0) \lor B(1)) \lor \dots            можно показать по индукции
\dots  \lor \lnot (\lnot B(k) \lor B(k')) \lor B(k')   (ослабление, перестановка, де-морган)
\lnot B(0) \lor \lnot (\forall x(\lnot B(x) \lor B(x'))) \lor B(k')    k раз квантификация, перестановки, сокращ.
Применим бесконечную индукцию относительно первого
и третьего терма и получим что надо.
\end{enumerate}

Окей, с аксиомами разобрались.
И еще есть два правила вывода
\begin{enumerate}
\item MP
B            условие
\lnot B \lor A       условие
A            сечение
\item GEN
B(x)         условие
Продвигаясь от этой формулы вверх можно поменять все
x на k, тогда верно
$$\vdash B(k)$$
На основании принципа бесконечной индукции доказываем \forall xB(x)
\item A \to B(t) => A \to \forall x.B(x)
\lnot A \lor B
Заменим все вхождения перменной в доказательстве в
ФА формулы \lnot A \lor B на 0, 1, 2\dots ,
тогда по бесконечной индукции:
$\lnot A \lor B(0), \lnot A \lor B(1), \dots  \vdash \lnot A \lor \forall x.B$   (только B(0) \lor \lnot A везде)
\item A(t) \to B => \exists x.A(x) \to B
$\lnot A \lor B \vdash \lnot \lnot \forall x.(\lnot A(x)) \to B$
Заменим все вхождения свободной перменной t в ФА на конкретные.
Получим счетное мн-во д-в \lnot A(0) \lor B, \lnot A(1) \lor B, \dots
по беск. индукции
\forall x.\lnot A(x) \lor B
\lnot \lnot \forall x.\lnot A(x) \lor B -  навесили двойное отрицание
\end{enumerate}
\subsubsection{Следствие: непротиворечивость S∞ влечет непротиворечивость S}
\label{sec-18-3-4}
Пусть в S доказуемо \lnot (0=0), тогда оно доказуемо и в
S∞, тогда
\uline{A\lor 0=0$_{\text{\lnot }}$(0=0)\lor A}          Аргументы получаются по ослаблению
\uline{A\lor A}
A
\subsection{Теорема Генцена об устранении сечений}
\label{sec-18-4}
\subsubsection{Лемма: сильные правила 2, 3, 5 обратимы}
\label{sec-18-4-1}
Правила обратимы и их д-во имеет порядок и степень не больше,
чем первоначальное
Нам дают доказательство формулы, мы строим новое дерево,
в котором из результата следуют посылки.
\begin{enumerate}
\item Правило Де-Моргана
\lnot (B \lor E) \lor D
Рассмотрим вывод формулы. Проследим вхождения
подформул \lnot (B \lor E)  формулы, которые соответствуют
вхождению в конечную формулу (то есть не те, которые
исчезают в сечении). Мы пройдем через всякий случай
применения слабого правила и сильные, когда \lnot (B \lor E)
является боковой формулой. Остановка произойдет
либо в ослаблении $( F \vdash \lnot (B \lor E) \lor F )$ либо в Де-Моргане
$( \lnot B \lor F, \lnot E \lor F \vdash \lnot (B \lor E) \lor F )$. Совокупность таких
вхождений формулы назовем ее \textbf{историей}.
Если все вхождения формулы \lnot (B \lor E) в ее истории заменить
на \lnot B, то в результате получится вывод \lnot B \lor D. Аналогично
можем вывести \lnot E \lor D.
\item Правило отрицания
\lnot \lnot B \lor D
Заменим все вхождения \lnot \lnot B в ее истории на B, получим
вывод B \lor D
\item Бесконечная индукция
\forall xB(x) \lor D
Заменим все вхождения \forall xB(x) в ее истории (1 элемент) на
B(k) (причем если мы уткнулись в бесконечную индукцию,
выберем только ту ветку, которая соответствует B(k))
Тогда для любого k получим вывод B(k)
\end{enumerate}
\subsubsection{Теорема: устранение сечения}
\label{sec-18-4-2}
Если для A в S∞ существует вывод (m, a), то существует
вывод в S∞ (n, 2ᵃ), где n < m
Докажем через трансфинитную индукцию по порядку a вывода A
\begin{itemize}
\item База: порядок вывода 0
Вывод не содержит сечений и его степень 0
\item Переход: пусть теорема верна для выводов порядков меньших a
Будем продвигаться вверх по выводу, пока не встретим первое
применение сильного правила или сечения.
\begin{enumerate}
\item Сильное правило
Пусть его посылки занумерованы порядковыми числами aₗ
Согласно индуктивному предположению, для этих посылок
существует дерево вывода F со степенью < m и порядком 2$^{\text{(a_i)}}$.
Заменим таким деревом то поддерево данного дерева вывода,
заключительной формулой которого служит рассматриваемое
вхождение F. Сделав так со всеми посылками мы получим новое
дерево для A, отнесем ему порядковое число 2ᵃ > 2$^{\text{(a_i)}}$
(пояснение от меня, потому что я чет долго доганял - каждую
посылку заменяем по индукции на ее модное новое дерево
от нее самой же).
\item Сечение
Значит имеем что-то на уровне
\uline{C\lor B\_$_{\text{\lnot B\lor D}}$}
C\lor D
Согласно индуктивному предположению для C\lor B и \lnot B\lor D существуют
выводы степеней меньших m и порядков 2$^{\text{a_1}}$, 2$^{\text{a₂}}$. Рассмотрим
разные случаи строения B:
\begin{enumerate}
\item B - это элементарная формула. Одна из формул B и \lnot B есть
аксиома. Пусть K - та, которая не аксиома.
По предположениию поддерево основного дерева вывода с K
может быть заменено другим со степенью n и порядка 2$^{\text{(a}_{\text{i}}\text{)}}$
(i = 1 или 2, смотря в какой посылке K).
В этом новом дереве расмотрим историю K, начальные формулы
в которой могут возникнуть только по ослаблению (а исчезать
по сечению). Поэтому удаление всех вхождений K из истории
приводит к построению дерева вывода для D или C порядка
2$^{\text{a_i}}$. Отсюда с помощью ослабления получаем дерево вывода
для C \lor D порядка 2$^{\text{a}}$. Степень меньше m (одно сечение убрали).
\item B - это \lnot E, тогда посылки выглядят как
C\lor \lnot E, \lnot \lnot E\lor D
Существует дерево вывода для \lnot \lnot E\lor D степени < m и порядка 2$^{\text{a₂}}$
В силу леммы об обратимости можно построить дерево вывода
E\lor D порядка 2$^{\text{a₂}}$ степени < m
Кроме того существует дерево вывода степени < m и порядка 2$^{\text{a_1}}$
для левой посылки, тогда построим из них новое дерево
вывода
\uline{E\lor D}   \uline{C\lor \lnot E}
\uline{D\lor E\_$_{\text{\lnot E\lor C}}$}
\uline{D\lor C}
C\lor D
Степень выделенного здесь сечения на единицу меньше общего
числа связок и кванторов в \lnot E, которое само по себе \le m.
Формуле C\lor D можно присвоить порядковое число C \lor D по свойству
посылок новог сечения.
\item B - это E\lor F, посылки:
C\lor E\lor F, \lnot (E\lor F)\lor D
Существует дерево вывода для правой посылки < m и порядка 2$^{\text{a₂}}$,
по лемме об обратимости существуют выводы степеней < m и порядка
2$^{\text{a₂}}$ для \lnot E\lor D и \lnot F\lor D (по Де-Моргану). По предположению индукции
есть еще дерево вывода для левой посылки.
Из последних трех построим такое:
\uline{C\lor E\lor F\_$_{\text{\lnot F\lor D}}$}
   \uline{C\lor E\lor D}
   \uline{C\lor D\lor E\_\_\_$_{\text{\lnot E\lor D}}$}
       \uline{C\lor D\lor D}
C\lor D
Степень сечения уменьшили (в каждом на 1), в формуле
C\lor E\lor D можно дать порядок $2^{\max(a_1, a₂)}$+1, а остальным - $2^a$
\item B - это \forall xE, посылки:
C\lor \forall xE  (\lnot \forall xE)\lor D
По индуктивному предположению для левой посылки можно построить
дерево вывода (< m,2$^{\text{a_1}}$). В силу указания в начале 2 леммы о
эквивалентности S∞ и ФА (если $\vdash A(t), t=s, то \vdash A(s)$) и леммы
об обратимости для любого постоянного терма z существует вывод
C\lor E(z).
Можем и правую посылку заменить (< m, 2$^{\text{a₂}}$) по инд.предположению
Тогда в правой посылке история \lnot (\forall xE) может начинаться либо
с ослабления либо с квантификации.
Заменим на C все такие начала - если это ослабление, то просто
подменим вместо \lnot (\forall xE) новое C, если квантификация, то
\uline{\lnot E(t)\lor F\_\_}         \uline{C\lor E(t)\_$_{\text{\lnot E}}$(t)\lor F}   //первое мы взяли из левой посылки
\lnot (\forall xE(x))\lor F   =>          C\lor F
В результате мы получили дерево вывода C\lor D
Степень меньше m, поскольку связку мы одну убрали
\textbf{тут еще ОЧЕНЬ АДОВЫЕ оценки порядка C\lor D, но он 2$^{\text{a}}$}
\end{enumerate}
\end{enumerate}
\end{itemize}
\subsubsection{Следствие: устранение всех сечений}
\label{sec-18-4-3}
Воспользуемся леммой об устранении сечения, пока степень вывода
не станет равна нулю. Тогда порядок будет что-то на уровне степенной
башни из двоек, в вершине которой первоначальная степень, а количество
двоек -- количество применения леммы.
\subsubsection{Следствие: S∞ непротивроечива}
\label{sec-18-4-4}
Если S∞ противоречива, в ней докажется $0\ne 0\lor 0\ne 0\lor \dotsb \lor 0\ne 0$.
В ее истории появление 0\ne 0 может быть только из-за ослабления,
но тогда посылкой ослабления тоже будет 0\ne 0. Если в д-ве есть
сечения, мы можем перестроить его, устранив все сечения, а затем провести
рассуждения вновь и увидеть, что такая формула недоказуема, а значит
S∞ непротиворечива, значит и ФА тоже непротиворечива.
\section{Ключевые фигуры}
\label{sec-19}
\begin{itemize}
\item Станислав Яськовски - 1906
Aлгебра Яськовского, один из первых исследователей ИИВ
\item Герхард Генцен - 1909
Теорема об устранении сечения (1935)
\item Курт Гёдель - 1906
Теоремы о непротиворечивости - 1930
Чуть ли не все остальное
\item Сол Крипке - 1940
Семантика Крипке - 1960-1970
\item Дэвид Гильберт - 1862, Пауль Бернайс - 1888
Основания математики - 1934, 39
\item Их достаточно много, а времени мало.
\end{itemize}

