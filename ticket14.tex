\section{Ticket 14: oрдиналы}
\label{sec-16}
\subsection{Ординальные числа}
\label{sec-16-1}
\begin{itemize}
\item Определение вполне упорядоченного множества (фундированное
с линейныи порядком).
\item Определение транзитивного множества:\\
Множество $X$ транзитивно, если
$\forall a \forall b((a \in b \land b \in x) \to a \in x)$
\item Ординал -- транзитивное вполне упорядоченное отношением $\in$ мн-во
\item Верхняя грань множества ординалов $S$:\\
$C | \{C = min(X) \land C \in X | X = \{z | \forall (y \in S)(z \geq y)\}\}$
$C = Upb(S)$
$Upb(\{\emptyset\}) = \{\emptyset\}$
\item $Successor ordinal$ (сакцессорный ординал?):\\
Это $b = a' = a \cup \{a\}$
\item Предельный ординал:\\
Ординал, не являющийся ни $0$ ни $successor$'ом.
\item Недостижимый ординал:\\
$\epsilon$ - такой ординал, что $\epsilon = w^{\epsilon}$\\
$\epsilon_0$ = $Upb(w, w^{w}$, $w^{w^{w}}$, $w^{w^{w^w}}$, \dots) -- минимальный из $\epsilon$ ординалов
\item Канторова форма -- форма вида $\sum(a*w^b+c)$, где $b$ -- ординал, последовательность строго убывает по $b$. Есть слабая канторова форма, где вместо $a$ ($a \in N$) пишут $a$ раз $w^b$. В канторовой форме приятно заниматься сложениями и прочим, потому что всякие $upb$ слишком ниочем.
\end{itemize}
\subsection{Операции над ординальными числами}
\label{sec-16-2}
\subsubsection{Стабилизация убывающей последовательности}
\label{sec-16-2-1}
Допустим, что есть убывающая последовательность ординалов $x_1, x_2, \dotsc$ Возьмем ординал $x_1 + 1 = x_0$. Тогда $\{x_1, x_2, \dots \} \in x_0$. $x_0$ не пусто, значит там есть минимальный элемент по определению порядка на ординале. Пусть этот элемент -- $m$. Тогда поскольку $m \in x_0$, то $m = x_i$ для какого-то $i$ нашей убывающей последовательности.
\begin{enumerate}
\item Последовательность убывает нестрого.\\
Тогда все $x_k \leq m$, для $k > i$. Это выполняется, если $x_k = x_i$, тогда последовательность стабилизируется в $m$.
\item Последовательность убывает строго.\\
Тогда все $x_k < m$ для $k > i$, но $m$ - минимум множества. Противоречие.\\
Убывающей строго последовательности ординалов не существует.
\end{enumerate}
\subsubsection{Арифметические операции через Upb}
\label{sec-16-2-2}
Пусть lim(a) = предельный ординал a
\begin{align*}
&x + 0 &= &x \\
&x + c' &= &(x + c)' \\
&x + lim(a) &= &Upb\{x + c | c < a\} \\
\\
&x * 0 &= &0 \\
&x * c' &= &x * c + x \\
&x * lim(a) &= &Upb\{x * c | c < a\} \\
\\
&x^{0} &= &1\\
&x^{c'} &= &x^c * x\\
&x^{lim(a)} &= &Upb\{x^c | c < a\}
\end{align*}

Ну вот короче можно так, только приходится много думать как реализовывать $Upb$. Или только у меня так.

$2^{w} = Upb(2, 4, 8, \dots ) = w$
\subsubsection{Арифметические операции через Канторову форму}
\label{sec-16-2-3}
Хорошо описано в этой статье:\\
\href{http://www.google.ru/url?sa=t&rct=j&q=&esrc=s&source=web&cd=1&ved=0CB4QFjAA&url=http://www.ccs.neu.edu/home/pete/pub/cade-algorithms-ordinal-arithmetic.pdf&ei=FDW6\lor JOYNuvXyQPd0ILQBQ&usg=AFQjCNENBOBOdKbbqBYN3iFhmAu_jFD2Sw&sig2=1UISFzJ_21I8f1YScX7Tkw&bvm=bv.83829542,d.bGQ&cad=rjt}{Algorithms for Ordinal Arithmetic}\\
Переписывать довольно громоздко, учитывая количество вспомогательных функций. Есть в гитхабе Михаила Волхова (\href{https://github.com/volhovM/mathlogic}{github}) реализованное

