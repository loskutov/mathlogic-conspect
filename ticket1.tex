\section{Исчисление высказываний}
\label{sec-3}
\subsection{Определения (исчисление, высказывание, оценкa\ldots{})}
\label{sec-3-1}
Формальная система с алгеброй Яськовского $J_{0}$ в качестве модели (множество
истинностных значений $\lbrace 0, 1 \rbrace$). Формальная теория нулевого порядка, кванторов
нет, предикаты - это пропозициональные переменные.
\subsection{Общезначимость, доказуемость, выводимость}
\label{sec-3-2}
\begin{itemize}
\item Формула называется общезначимой в теории с моделью если эта формула
верна в любой модели данной теории. Например в ИВ формула называется
общезначимой если любая оценка формулы на любых значениях пропозициональных
переменных (что является моделью ИВ) возвращает истину. В ИИВ же общезначимостью
называется существование формулы во всех мирах всех возможных моделей крипке
(которые являются моделью для ИИВ).
\item Доказуемость - свойство формулы в теории, значащее, что существует
доказательство для этой формулы. Доказательство для теории тоже определяется
по разному (последовательность утверждений, каждое из которых есть аксиома
или следует по правилу вывода из предыдущих в ИВ, дерево с выводами в $S\infty$)
\end{itemize}
\subsection{Схемы аксиом и правило вывода Modus Ponens}
\label{sec-3-3}
Схемы аксиом:
\begin{enumerate}
\item $\alpha \to \beta \to \alpha$
\item $(\alpha \to \beta) \to (\alpha \to \beta \to \gamma) \to (\alpha \to \gamma)$
\item $\alpha \to \beta \to \alpha \land \beta$
\item $\alpha \land \beta \to \alpha$
\item $\alpha \land \beta \to \beta$
\item $\alpha \to \alpha \lor \beta$
\item $\beta \to \alpha \lor \beta$
\item $(\alpha \to \beta) \to (\gamma \to \beta) \to (\alpha \lor \gamma \to \beta)$
\item $(\alpha \to \beta) \to (\alpha \to \lnot \beta) \to \lnot \alpha$
\item $\lnot \lnot \alpha \to \alpha$
\end{enumerate}

Единственное правило вывода Modus Ponens (M.P.):
\[\infer{\beta}{\alpha & (\alpha \rightarrow \beta)}\]
\subsection{Теорема о дедукции}
\label{sec-3-4}
\begin{theorem}
	$\Gamma, \alpha \vdash \beta \Leftrightarrow \Gamma \vdash \alpha \to \beta$
\end{theorem}
\begin{proof}
\leavevmode
$\Rightarrow$)
Нужно переместить последнее предположение вправо.
Будем доказывать индукцией по доказательству. Обработаем три случая в переходе:
\begin{enumerate}
\item Аксиома или предположение \\
$A$ \\
$A\to \alpha \to A$ \\
$\alpha \to A$
\item Modus Ponens \\
По предположению индукцию уже было доказано $\alpha \to A$, $\alpha \to A \to B$ \\
$(\alpha \to A)\to (\alpha \to A \to B)\to (\alpha \to B)$ \\
$(\alpha \to A\to B)\to (a\to B)$ \\
$\alpha \to B$
\item Само это выражение \\
$\alpha\to \alpha$ умеем доказывать
\end{enumerate}
$\Leftarrow$) Если нужно переместить влево, то перемещаем и добавляем следующее: \\
$A\to B$ (последнее) \\
$A$    (перемещенное) \\
$B$
\end{proof}

\subsection{Корректность исчисления высказываний относительно алгебры Яськовского}
\label{sec-3-5}
\begin{itemize}
\item Индукцией по доказательству -- если аксиома, то она
тавтология, все ок. Если модус поненс, то таблица
истинности для импликации и все ок
\end{itemize}

%%% Local Variables:
%%% mode: latex
%%% TeX-master: "mathlogic"
%%% TeX-engine: xetex
%%% End: