% Created 2015-01-21 Ср 21:06
%\documentclass[draft,12pt]{article}
\documentclass[fleqn,12pt]{article}
\usepackage{euler}
\usepackage{amsmath}
\usepackage{epigraph}
\usepackage{mathtools}
\usepackage{amssymb}
\usepackage{proof} % M. P. и его друзья
\usepackage[linkbordercolor={0 0 1}, pdftitle={Курс матлогики по Штукенбергу Д. Г.}]{hyperref}
\usepackage{amsthm}
\usepackage{comment}
\usepackage{stmaryrd} % ⟦⟧
\usepackage{clrscode3e}
\usepackage{enumitem} % всякие модные enumerate-ы (latin letters, roman numbers etc)
\usepackage{xcolor} % red color for worries
\usepackage{pict2e} % templates
\usepackage[left=2cm,right=1.6cm,top=2cm,bottom=2cm,bindingoffset=0cm]{geometry}
\usepackage{microtype} % модная микротипографика, нифига не работающая с XeLaTeX (ну, почти)
\usepackage{fontspec}
\setmainfont[Ligatures=TeX]{Palatino Linotype}
\setmonofont{DejaVu Sans Mono}
\usepackage{polyglossia}
\setdefaultlanguage[babelshorthands=true]{russian}

%\hyphenpenalty=5000


\renewcommand{\le}{\leqslant} % ⩽
\renewcommand{\leq}{\leqslant} % ⩽
\renewcommand{\ge}{\geqslant} % ⩾
\renewcommand{\geq}{\geqslant} % ⩾
\renewcommand{\phi}{\varphi}
\renewcommand{\epsilon}{\varepsilon}

% Declaring some new operators
\DeclareMathOperator{\plog}{plog}
\DeclareMathOperator{\Int}{Int}
\DeclareMathOperator{\Proof}{Proof}
\renewcommand{\land}{\mathbin{\&}}
\newcommand{\perc}{\mathbin{\%}}

\newcommand{\ltemplate}{\begin{picture}(5,7)
\put(.5,4){\line(1,-1){5}}
\put(.2,4.1){\line(1,-1){5.2}} % just to make it more bold lol
\put(.5,4){\line(1,1){5}}
\put(.2,3.9){\line(1,1){5.2}} % just to make it more bold lol
\end{picture}}
\newcommand{\rtemplate}{\begin{picture}(5,7)
\put(4.5,4){\line(-1,-1){5}}
\put(4.8,4.1){\line(-1,-1){5.2}} % just to make it more bold lol
\put(4.5,4){\line(-1,1){5}}
\put(4.8,3.9){\line(-1,1){5.2}} % just to make it more bold lol
\end{picture}}
\newcommand{\template}[1]{\ltemplate #1\rtemplate}

\newcommand{\starsection}[1]{\addtocounter{section}{1}\setcounter{subsection}{0}%
    \addcontentsline{toc}{section}{\hspace{-2em}\texorpdfstring{$\star\quad$\thesection\hspace{1em}}{}#1}%
{\vspace{1.5em}\par\noindent\bfseries\Large \thesection$^\star$\hspace{0.5em} #1}\par\vspace{1em}\noindent}

% some magic for Godel numerals
\newbox\gnBoxA
\newdimen\gnCornerHgt
\setbox\gnBoxA=\hbox{$\ulcorner$}
\global\gnCornerHgt=\ht\gnBoxA
\newdimen\gnArgHgt
\def\Godel #1{%
\setbox\gnBoxA=\hbox{$#1$}%
\gnArgHgt=\ht\gnBoxA%
\ifnum     \gnArgHgt<\gnCornerHgt \gnArgHgt=0pt%
\else \advance \gnArgHgt by -\gnCornerHgt%
\fi \raise\gnArgHgt\hbox{$\ulcorner$} \box\gnBoxA %
\raise\gnArgHgt\hbox{$\urcorner$}}

\newcommand{\defeq}{\coloneqq}
\newcommand{\s}[1]{\texttt{#1}}
\newcommand{\xl}{$\lambda$}
\newcommand{\+}{\lambda}
\newcommand{\bredmath}{\ \longrightarrow_\beta\ }
\newcommand{\bred}{$\bredmath$}
\newcommand{\mbred}{$\ \longrightarrow\!\!\!\!\rightarrow_\beta\ $}
\newcommand{\lid}[1]{\textit{#1}}
\newcommand{\concat}{\hat{\ \ }}

\newcommand\myworries[1]{\textcolor{red}{#1}}

\def\ra{\rightarrow}

\tolerance 1000 % чтобы не очковал переносить

\renewcommand{\theenumii}{\asbuk{enumii}}
\AddEnumerateCounter{\asbuk}{\@asbuk}{ы}

\DeclareRobustCommand{\divby}{%
  \mathrel{\vbox{\baselineskip.65ex\lineskiplimit0pt\hbox{.}\hbox{.}\hbox{.}}}%
}

\author{Daniyar Itegulov, Aleksei Latyshev, Ignat Loskutov}
\date{\today}
\title{Курс математической логики по Штукенбергу Д.~Г.}

\begin{document}
\theoremstyle{definition}
\newtheorem*{definition}{Определение}%[section]
\newtheorem*{example}{Пример}
%\theoremstyle{theorem}
\newtheorem{theorem}{Теорема}[section]
\newtheorem{axiom}{Аксиома}[section]
\newtheorem{lemma}[theorem]{Лемма}

\maketitle
\setcounter{tocdepth}{2}
\tableofcontents

%Должно отображаться корректно: $x_1$,x₂x₃x_n, θ, \exists , ∑, ∉
\epigraph{%
    лан, всё не книжку верстаем))))000}
  {некто Игнат Лоскутов о качестве вёрстки}

Mykhail Volkhov, 2538, 2014Sep-2015Jan\\
Я не отвечаю за верность написанного - много информации
я придумал сам, много достал из недостоверных источников.
\include{base}
\include{defenitions}
\section{Исчисление высказываний}
\label{sec-3}
\subsection{Определения (исчисление, высказывание, оценкa\ldots{})}
\label{sec-3-1}
Формальная система с алгеброй Яськовского $J_{0}$ в качестве модели (множество
истинностных значений $\lbrace 0, 1 \rbrace$). Формальная теория нулевого порядка, кванторов
нет, предикаты - это пропозициональные переменные.
\subsection{Общезначимость, доказуемость, выводимость}
\label{sec-3-2}
\begin{itemize}
\item Формула называется общезначимой в теории с моделью если эта формула
верна в любой модели данной теории. Например в ИВ формула называется
общезначимой если любая оценка формулы на любых значениях пропозициональных
переменных (что является моделью ИВ) возвращает истину. В ИИВ же общезначимостью
называется существование формулы во всех мирах всех возможных моделей крипке
(которые являются моделью для ИИВ).
\item Доказуемость - свойство формулы в теории, значащее, что существует
доказательство для этой формулы. Доказательство для теории тоже определяется
по разному (последовательность утверждений, каждое из которых есть аксиома
или следует по правилу вывода из предыдущих в ИВ, дерево с выводами в $S\infty$)
\end{itemize}
\subsection{Схемы аксиом и правило вывода Modus Ponens}
\label{sec-3-3}
Схемы аксиом:
\begin{enumerate}
\item $\alpha \to \beta \to \alpha$
\item $(\alpha \to \beta) \to (\alpha \to \beta \to \gamma) \to (\alpha \to \gamma)$
\item $\alpha \to \beta \to \alpha \land \beta$
\item $\alpha \land \beta \to \alpha$
\item $\alpha \land \beta \to \beta$
\item $\alpha \to \alpha \lor \beta$
\item $\beta \to \alpha \lor \beta$
\item $(\alpha \to \beta) \to (\gamma \to \beta) \to (\alpha \lor \gamma \to \beta)$
\item $(\alpha \to \beta) \to (\alpha \to \lnot \beta) \to \lnot \alpha$
\item $\lnot \lnot \alpha \to \alpha$
\end{enumerate}

Единственное правило вывода Modus Ponens (M.P.):
\[\infer{\beta}{\alpha & (\alpha \rightarrow \beta)}\]
\subsection{Теорема о дедукции}
\label{sec-3-4}
\begin{theorem}
	$\Gamma, \alpha \vdash \beta \Leftrightarrow \Gamma \vdash \alpha \to \beta$
\end{theorem}
\begin{proof}
\leavevmode
$\Rightarrow$)
Нужно переместить последнее предположение вправо.
Будем доказывать индукцией по доказательству. Обработаем три случая в переходе:
\begin{enumerate}
\item Аксиома или предположение \\
$A$ \\
$A\to \alpha \to A$ \\
$\alpha \to A$
\item Modus Ponens \\
По предположению индукцию уже было доказано $\alpha \to A$, $\alpha \to A \to B$ \\
$(\alpha \to A)\to (\alpha \to A \to B)\to (\alpha \to B)$ \\
$(\alpha \to A\to B)\to (a\to B)$ \\
$\alpha \to B$
\item Само это выражение \\
$\alpha\to \alpha$ умеем доказывать
\end{enumerate}
$\Leftarrow$) Если нужно переместить влево, то перемещаем и добавляем следующее: \\
$A\to B$ (последнее) \\
$A$    (перемещенное) \\
$B$
\end{proof}

\subsection{Корректность исчисления высказываний относительно алгебры Яськовского}
\label{sec-3-5}
\begin{itemize}
\item Индукцией по доказательству -- если аксиома, то она
тавтология, все ок. Если модус поненс, то таблица
истинности для импликации и все ок
\end{itemize}

%%% Local Variables:
%%% mode: latex
%%% TeX-master: "mathlogic"
%%% TeX-engine: xetex
%%% End:
\include{ticket2}
\include{ticket3}
\include{ticket4}
\section{Логика второго порядка}
\label{sec-7}
\subsection{Основные определения}
\label{sec-7-1}
Смотрим коснпект ДГ
\subsection{Теорема о дедукции}
\label{sec-7-2}
\begin{theorem}
Если $\Gamma, \alpha \vdash \beta$, и в доказательстве отсутствуют применения правил для кванторов, использующих свободные переменные  из формулы $\alpha$, то $\Gamma \vdash \alpha \rightarrow \beta$
\end{theorem}
\begin{proof}
Будем рассматривать формулы в порядке сверху вниз. На $i$-ой строке встретили формулу $\delta_{i}$. Тогда докажем, что $\alpha \rightarrow \delta_{i}$. Разберем случаи:
\begin{enumerate}
\item $\delta_{i}$ - старая аксиома, совпадает с $\alpha$ или выводится по правилу M.P.\\
Тогда мы знаем, что делать из Теоремы о дедукции для ИВ
\item $\delta_{i}$ - новая аксиома\\
Тогда все то же самое, что и в старой аксиоме, но нужно так же проверить условие.
\item $\exists x (\psi) \rightarrow \phi$ - новое правило вывода
\begin{itemize}
\item Докажем вспомогательную лемму:
\begin{lemma}
$(\alpha \rightarrow (\beta \rightarrow \gamma)) \rightarrow (\beta \rightarrow (\alpha \rightarrow \gamma))$
\end{lemma}
\begin{proof}
Докажем, что $\alpha \rightarrow (\beta \rightarrow \gamma), \beta, \alpha \vdash \gamma$:\\
\begin{tabular}{lll}
(1) & $\alpha \rightarrow \beta \rightarrow \gamma$& Допущение\\
(2) & $\alpha$& Допущение\\
(3) & $\beta \rightarrow \gamma$& M.P. 2,1\\
(4) & $\beta$& Допущение\\
(5) & $\gamma$& M.P. 4,3\\
\end{tabular}
\end{proof}
\item По индукционному преположению мы знаем, что $\alpha \rightarrow \psi \rightarrow \phi$. Тогда докажем, что $\alpha \rightarrow \psi \rightarrow \phi, (\alpha \rightarrow \psi \rightarrow \phi) \rightarrow (\psi \rightarrow \alpha \rightarrow \phi) \vdash \alpha \rightarrow \exists x (\psi) \rightarrow \phi$:\\
\begin{tabular}{lll}
(1) & $(\alpha \rightarrow \psi \rightarrow \phi) \rightarrow (\psi \rightarrow \alpha \rightarrow \phi)$& Допущение\\
(2) & $\alpha \rightarrow \psi \rightarrow \phi$& Допущение\\
(3) & $\psi \rightarrow \alpha \rightarrow \phi$& M.P. 2,1\\
(4) & $\exists x (\psi) \rightarrow \alpha \rightarrow \phi$& Правило вывода 1\\
(5) & $(\exists x (\psi) \rightarrow \alpha \rightarrow \phi) \rightarrow (\alpha \rightarrow \exists x (\psi) \rightarrow \phi)$& Допущение\\
(6) & $\alpha \rightarrow \exists x (\psi) \rightarrow \phi$& M.P. 4,5\\
\end{tabular}
\end{itemize}
\item $\phi \rightarrow \forall x (\psi)$ - новое правило вывода
\begin{itemize}
\item Докажем вспомогательную лемму 1
\begin{lemma}
$(\alpha \land \beta \rightarrow \gamma) \rightarrow (\alpha \rightarrow \beta \rightarrow \gamma)$
\end{lemma}
\begin{proof}
Докажем, что $(\alpha \land \beta \rightarrow \gamma), \alpha, \beta \vdash \gamma$:\\
\begin{tabular}{lll}
(1) & $\alpha$& Допущение\\
(2) & $\beta$& Допущение\\
(3) & $\alpha \rightarrow \beta \rightarrow \alpha \land \beta$& Сх. акс. 1\\
(4) & $\beta \rightarrow \alpha \land \beta$ M.P. 1,3\\
(5) & $\alpha \land \beta$& M.P. 2,4\\
(6) & $\alpha \land \beta \rightarrow \gamma$& Допущение\\
(7) & $\gamma$& M.P. 5,6\\
\end{tabular}
\end{proof}
\item Докажем вспомогателньую лемму 2
\begin{lemma}\label{impltoand}
$(\alpha \rightarrow \beta \rightarrow \gamma) \rightarrow (\alpha \land \beta \rightarrow \gamma)$
\end{lemma}
\begin{proof}
Докажем, что $\alpha \rightarrow \beta \rightarrow \gamma, \alpha \land \beta \vdash \gamma$:\\
\begin{tabular}{lll}
(1) & $\alpha \land \beta \rightarrow \alpha$& Сх. акс. 4\\
(2) & $\alpha \land \beta$& Допущение\\
(3) & $\alpha$& M.P. 2,1\\
(4) & $\alpha \land \beta \rightarrow \beta$& Сх. акс. 5\\
(5) & $\beta$& M.P. 2,4\\
(6) & $\alpha \rightarrow \beta \rightarrow \gamma$& Допущение\\
(7) & $\beta \rightarrow \gamma$& M.P. 3,6\\
(8) & $\gamma$& M.P. 5,7\\
\end{tabular}
\end{proof}
\item По индукционному предположению мы знаем, что $\alpha \rightarrow \psi \rightarrow \phi$. Тогда докажем, что $\alpha \rightarrow \psi \rightarrow \phi \vdash \alpha \rightarrow \psi \rightarrow \forall (\phi)$.\\
\begin{tabular}{lll}
(1) & $(\alpha \rightarrow \psi \rightarrow \phi) \rightarrow (\alpha \land \psi \rightarrow \phi)$& Вспомогательная лемма 1\\
(2) & $\alpha \rightarrow \psi \rightarrow \phi$& Допущение\\
(3) & $\alpha \land \psi \rightarrow \phi$& M.P. 2,1\\
(4) & $\alpha \land \psi \rightarrow \forall (\phi)$& Правило вывода 2\\
(5) & $(\alpha \land \psi \rightarrow \forall (\phi)) \rightarrow (\alpha \rightarrow \psi \rightarrow \forall (\phi))$& Вспомогательная лемма 2\\
(6) & $\alpha \rightarrow \psi \rightarrow \forall (\phi)$& M.P. 4,5\\
\end{tabular}
\end{itemize}
\end{enumerate}
\end{proof}
\subsection{Корректность исчисления предикатов}
\label{sec-7-3}
Смотрим конспект ДГ

\include{ticket6}
\include{ticket7}
\include{ticket8}
\section{Ticket 9: представимость}
\label{sec-11}
\subsection{Функции, их представимость}
\label{sec-11-1}
Арифметическая функция -- это отображение $f: N_0^n \to N_0$ \\
Арифметическое отношение -- это $P \in N_0^n$ \\
Если $k \in N_0$, то $\overline k = 0'''''^\cdots$, где количество штрихов есть $k$.
\begin{itemize}
\item Арифметическое отношение $R \in N_0^n$ выразимо в ФА, если\\
$\exists a$ с $n$ свободными переменными:
$a(x_1,\dotsc, x_n)$, такая что
\begin{enumerate}
\item Eсли $R(k_1,\dotsc, k_n)$, то $\vdash a(\overline{k_1}, \dotsc, \overline{k_n})$
\item Eсли $\lnot R(k_1, \dotsc, k_n)$, то $\vdash \lnot a(\overline{k_1}, \dotsc, \overline{k_n})$
\end{enumerate}
\item $C_R$ - функция, равная $1$, если $R$, и равная $0$, если $\lnot R$
\item $\exists !y.\phi(y) = \exists y.\phi(y) \& \forall a\forall b(\phi(a) \& \phi(b) \to a = b)$
\item $f: N_0^n \to N_0$ представима в ФА, если $\exists a(x_1\ldots{}x_{n+1})$, что
$\forall x_1\dotsc x_{n+1}:$
\begin{enumerate}
\item $f(x_1, \dotsc, x_n) = x_{n+1} \Leftrightarrow \vdash a(\overline{x_1}, \dotsc, \overline{x_{n+1}})$
\item $\exists !b(a(\overline{x_1}, \dotsc, \overline{x_n}, b))$
\end{enumerate}
\end{itemize}
\subsection{Теорема о связи представимости и выразимости}
\label{sec-11-2}
\begin{theorem}
$R$ выразимо $\Leftrightarrow$ $C_r$ представимо
\end{theorem}
\begin{proof}
$\Rightarrow$ $a$ выражает $R$\\
$(a \to (x_n₊_1=0')) \& (\lnot a \to (x_n₊_1=0))$
представляет $C_r$\\
По выразимости $R \vdash a$; тогда $⊤\to a\to ⊤ => a\to ⊤$\\
По 10i, перенесенной к нам $a \to (\lnot a \to ⊥)$\\
правило с единственностью вроде понятно (хотя руками помахал, да)

$\Leftarrow$ $C_r \text{ представимо} \to R \text{ выразимо}$
Пусть представлять $C_r$ будет
$a(x_1, \dotsc, x_n,x_{n+1})$
Тогда определим, какая формула выражает $R$:
$a(\dotsc, 1)$
Из представимости:
\begin{itemize}
\item $\exists b.a(x_1\ldots{}x_{n+1})$
\item $\forall x\forall y(a(\ldots{}x) \& a(\ldots{} y) \to x = y)$
\item если $C_r(x_1\ldots{}x_n) = 1$, то $\vdash a(x_1\ldots{}x_n,1)$
\item если $C_r(\dots) = 0$, то $\vdash a(\dotsc, 0)$
\end{itemize}
Докажем выводимость
\begin{enumerate}
\item Покажем, что если $R(x_1\ldots{}x_n)$, то $\vdash a(x_1\ldots{}x_n, 1)$
Из представимости прямо ровно.
\item Покажем, что если $\lnot R(x_1\ldots{}.x_n), то \vdash \lnot a(x_1\ldots{}x_n, 1)$\\
По единственности

$\forall x\forall y(a(x_1\ldots{}x_n, x) \& a(x_1\ldots{}x_n, y) \to x = y)$\\
$a(x_1\ldots{}x_n, 0) \& a(x_1\ldots{}x_n, 1) \to (0 = 1)$ (спустя две акс. и 2 MP)\\
Делаем дедукцию\\
$a(x_1\ldots{}x_n, 0) \& a(x_1\ldots{}x_n, 1) \vdash ⊥$\\
$a(x_1\ldots{}x_n, 0) \& a(x_1\ldots{}x_n, 1) \to a(x_1\ldots{}x_n, 0)$\\
$a(x_1\ldots{}x_n, 0)$\\
$\lnot a(x_1\ldots{}x_n, 0)$ по представимости
$a(x_1\ldots{}x_n, 0) \to (\lnot a(x_1\ldots{}x_n, 0) \to \lnot a(x_1\ldots{}x_n, 1))$ (10i в ИИВ, доказуема в предикатах)\\
$\lnot a(x_1\ldots{}x_n, 1)$\\
Хотим $\lnot a(x_1\ldots{}x_n, 1)$
\end{enumerate}
\end{proof}
\subsection{beta-функция Гёделя, китайская теорема об остатках}
\label{sec-11-3}
$\beta(b, c, i) = b \% (1 + c * (1 + i))$\\
Где $\%(a, b) = d$, что $\forall m . (d + m * b = a), m \geq 0, 0 \leq d \leq b$

\subsubsection{Китайская теорема об остатках}
\label{sec-11-3-1}
\begin{theorem}
$n_1\ldots{}n_k$ - попарно взаимно простые целый числа\\
$r_1\ldots{}r_k$ - любые целые числа, что $0 \leq r_1 < n_1$\\
Тогда: $\exists b \forall i  r_1 = b \% n_k$
\end{theorem}
\begin{proof}
Без доказательства
\end{proof}

\subsubsection{Гёделева Г-последовательность}
\label{sec-11-3-2}
$\Gamma_1 = (i + 1) * c + 1$\\
$\Gamma(c) = 1 * c + 1, 2 * c + 1, 3 * c + 1, \ldots (n + 1) * c + 1$
\begin{theorem}
$\Gamma(c)$ подходит на роль $n_1 \ldots n_k$ в китайской теореме об остатках
\end{theorem}
\begin{proof}
Выделим последовательность размера $n$: $k_1 \ldots k_n$.\\
Чтобы это выполнялось возьмем $c = (max(k_1\ldots{}k_n))!$
\begin{enumerate}
\item В $\Gamma$ любые два элемента попарно взаимно простые
\label{sec-11-3-2-1}
Пусть $\Gamma_1 \vdots \Gamma_j$ имеют общий делитель $p > 1$. Мы можем его разложить на простые множители и взять какой-нибудь простой (любое число раскладывается на простые множители).\\
Тогда $(\Gamma_1 - \Gamma_j) \divby p, (c * (i - j)) \divby p$. Заметим, что $\lnot (c \divby p)$, потому что иначе $\Gamma_1 = 1 + c * (i + 1) \divby p$ и $c * (i + 1) \divby p$, а они отличаются на единицу. Тогда $(i - j) \divby p$, но $c = m! m > n$, а $i - j < n$, значит $c \divby p$.
\item Каждое $k_1 < Г_1$
\label{sec-11-3-2-2}
$k_1 \leq c < 1 + c * (i + 1) = \Gamma_1$
\end{enumerate}
\end{proof}
\begin{comment}
%%%%%%%%%%%%%%%%%%%%%%%%%%%%%%%%%%%%%%%%%%%%%%%%%%%%%%%%%%%%%%%%%%%%%%%%%%%%%%%%%%%%%%%%%%
\subsubsection{Лемма о β-функции}
\label{sec-11-3-3}
Увидим, что β(b,c,i) считает остаток от деления b на
(i + 1) * c + 1 - элемент Геделевой последовательности.
\begin{itemize}
\item <a_0\ldots{}a_n>\in N \to \exists b\exists c(aₖ=β(b,i,c)) - β-функция кодирует
последовательность натуральных чисел и может
доставать по индексу i

a_0\ldots{}a_n  - последовательность натуральных чисел
тогда существует такое c, что Г = 1*c+1, 2*c+1,\ldots{}
если c ≥ max(a_0\ldots{}a_n), то aₖ < (i+1)*c+1
Но по свойству Г элементы попарно взаимно просты
тогда сравнения
a_0 \% (0+1)*c+1
a_1 \% (1+1)*c+1
\ldots{}\ldots{}\ldots{}\ldots{}.
a_n \% (n+1)*c+1
имеют общее решение b по китайской теореме об остатках
тогда aᵢ = b \% (i+1)*c + 1
но это и есть β-функция
aᵢ = β(b,c,i)
\end{itemize}
\subsubsection{Представимость β-функции Гёделя в ФА}
\label{sec-11-3-4}
β-функция представима в ФА отношением
B(b,c,i,d) = \exists q((b = q * (1 + c * (i + 1)) + d) \& (d < 1 + c * (i + 1)))
Пусть 1 + c * (i + 1) = z
Докажем условия представимости:
\begin{enumerate}
\item Эквивалентность
\begin{enumerate}
\item β(b,c,i) = d, тогда $\vdash B(b,c,i,d)$
b = z * (1 + c * (i + 1))    (это и следующее - из леммы о β) P
d < 1 + c * (i + 1)                                           Q
P \to Q \to P \& Q
P \& Q
P \& Q \to \exists q.(P \& Q) [z:= q]
\exists q.(P \& Q)
\item Пусть $\vdash B(b,c,i,d)$, тогда
\exists q.(P \& Q)
подберем такое q (по лемме)
P \& Q \to P
P \& Q \to Q
P
Q
значит β(b,c,i) = d
\end{enumerate}
\item Единственность
Следует из леммы.
\end{enumerate}
\subsection{Теорема о представимости рекурсивных функций Z, N, U}
\label{sec-11-4}
\begin{enumerate}
\item Z
Z(a, b) = (b = 0)
\begin{itemize}
\item Z(a) = b верно, тогда b = 0
b = 0
\item (b = 0)
b = 0
тогда Z(0) = 0, все ок
\item \exists y.\phi(y) \& \forall a\forall b(\phi(a) \& \phi(b) \to a = b)
Тоже как-то несложно
\end{itemize}
\item N
N(a, b) = (a = b')
\begin{itemize}
\item N(a) = b, тогда a = b'
a = b'
\item a = b', тогда
N(a) = b
\item Третье не хочу
\end{itemize}
\item U_nᵢ
U_nᵢ(x_1\ldots{}x_n) = (x_1 = x_1) \& (x₂ = x₂) \& \ldots{} \& (x_n₊_1 = xᵢ)
\begin{itemize}
\item U(..) = xᵢ, тогда x_n₊_1 = x_1
x_1 = x_1 доказывается
\ldots{}
x_n = x_n доказывается
x_n₊_1 = x_1 по условию
объединяем все \&
\item (x_1 = x_1) \& \ldots{}.
вытаскиваем каждый элемент и тогда видим, что
проекция делает ровно то, что должна.
\item \exists q.(x_n₊_1 = q)
ХЗ
\item \forall a\forall b(x(\ldots{}a)\&x(\ldots{}.b) \to a = b)
Для конкретных a, b обявляем a = b - ⊤, тогда выводим
из него конъюнкцию и навешиваем два квантора
\end{itemize}
\end{enumerate}
\subsection{Теорема о представимости S}
\label{sec-11-5}
Eсли f и g_1\ldots{}g_n представимы, то S<f, g_1\ldots{}g_n> представима
Пусть F, G_1\ldots{}G_n представляют их.
S(a_1\ldots{}aₘ, b) = \exists b_1\ldots{}\exists b_n(G_1(a_1\ldots{}a_n, b_1) \& \ldots{} \& Gn(a_1\ldots{}aₘ, b_n)
\& F(b_1\ldots{}b_n, b))
\begin{itemize}
\item Пусть S(a_1\ldots{}a_n) = b, тогда существуют такие b_1\ldots{}b_n, что \#каждый аргумент\#
Поскольку f,g_1\ldots{}g_n представимы, то доказуемы по представимости
f(b_1\ldots{}b_n, b)
g_1(a_1\ldots{}a_n, b_1)
\ldots{}
g_n(a_1\ldots{}a_n, b_n)
g_1 \& g₂ \& \ldots{} \& g_n \& f    объединили \&     "P"
"P" \to \exists b_1."P[b_1:=b_1]" + MP
\ldots{}
Ну и навесили кванторы, да.
\item Пусть верна формула с кванторами. Тогда она и есть уже то, что надо
\item не могу, да и вообще нигде это свойство не доказывается
\end{itemize}
\subsection{Теорема о представимости R}
\label{sec-11-6}
Пусть f, g представимы F, G. Тогда R<f,g> представима.
f: N^n\to N, g:N^n⁺²\to N
r(x_1\ldots{}x_n, k, a) =
   \exists b\exists c(
        \exists k(β(b, c, 0, k) \& \phi(x_1\ldots{}x_n, k))
        \& B(b, c, x_n₊_1, a)
        \& \forall k(k<x_n₊_1 \to \exists d\exists e(B(b,c,k,d)\&B(b,c,k',e)\&G(x_1\ldots{}x_n,k,d,e))))
Единственная возможность осознать -- внимательно прочесть формулу.
Тут β-функция используется в качестве функии отображения нашего шага
вычисления рекурсии в результат, типа
0 - F(\ldots{})
1 - G(\ldots{})
\ldots{}
n - G(\ldots{})
\subsection{Теорема о представимости μ}
\label{sec-11-7}
f: N^{n+1}\to N представима F, тогда μ<f> представима M
Μ<F>(x_1\ldots{}x_n₊_1) = F(x_1\ldots{}x_n, x_n₊_1, 0) \& \forall y((y < x_n₊_1) \to \lnot F(x_1\ldots{}x_n, y, 0))
\begin{itemize}
\item μ<f>(x_1\ldots{}x_n) = x_n₊_1, тогда x_n₊_1 - минимальное k, такое что f(x_1..x_n,k) = 0
то есть имеем
F(x_1\ldots{}x_n, x_n₊_1, 0)
\forall x.(k < x \to \lnot F(x_1\ldots{}x_n, k, 0))
Просто объединим конъюнкцией
\item обратно ей же и разъединим
\end{itemize}
\end{comment}

\section{Ticket 10: Тьюринг}
\myworries{Нет в билетах, но на лекциях было}
\label{sec-12}
\subsection{Арифметические отношения, их выразимость}
\label{sec-12-1}
\item Арифметическое отношение $R \in N_0^n$ выразимо в ФА, если $\exists a$ с $n$ свободными переменными $a(x_1, \dotsc, x_n)$, такая что:
\begin{enumerate}
\item Eсли $R(k_1\dots k_n)$, то $\vdash a(\overline{k_1}\dots \overline{k_n})$
\item Eсли $\lnot R(k_1..k_n)$, то $\vdash \lnot a(\overline{k_1} \dots \overline{k_n})$
\end{enumerate}
\subsection{Гёделева нумерация}
\label{sec-12-2}
\begin{center}
\begin{tabular}{lrl}
$a$ & $\Godel{a}$ & описание\\
\hline
$($        & $3$ & \\
$)$        & $5$ & \\
$,$        & $7$ & \\
$\lnot$    & $9$ & \\
$\to$      & $11$ & \\
$\lor$     & $13$ & \\
$\&$       & $15$ & \\
$\forall$  & $17$ & \\
$\exists$  & $19$ & \\
$x_k$      & $21 + 6 \cdot k$ & переменные\\
$f^n_k$    & $23 + 6 \cdot 2^k \cdot 3^n$ & n-местные функцион. символы (', +, *)\\
$P^n_k$    & $25 + 6 \cdot 2^k \cdot 3^n$ & n-местные предикаты (=)\\
\hline
\end{tabular}
\end{center}
Последовательность значков будем составлять так:\\
$a_1, \dotsc, a_n$ - наши простые числа, соответствующие символам, тогда $p_1^{a_1} * p_2^{a_2}, \dotsc, p_n^{a_n}$ - геделев нумерал строки, составленной из символов.

Если $a$ -- выражение, то $\Godel{a}$ - выражение в Геделевой форме. Тогда если $a$ -- выражение, $\overline{\Godel{a}}$ - это элемент предметного множества ФА, соответствующий нолику с количеством черточек, равным $\Godel{a}$.

Доказательство -- это последовательность простых чисел, возведенная
в геделевы нумералы выражений, являющихся составляющими док-ва, по
порядку. Аналогично с составлением строки из символов.

Тогда определим следующие операции с нумералами:
\begin{itemize}
\item $\plog(a, b) = \max n : a \perc b^n = 0$\\
Иногда вместо $b$ стоит $P_{b}$, где $P_{b}$ -- простое число с индексом $b$.\\
Функция берет геделев нумерал и достает у него $i$-й элемент последовательности
\item $len = \max n : a \perc p_n$\\
Возвращает длину строки доказательства
\item $s@t = p_1^{\plog(s, 1)} * \dots  * p_{len(s)}^{\plog(s, len(s))} *
p_{len(s+1)}^{\plog(t, 1)} * \dots  * p_{len(s)_len(t)}^{\plog(t, len(t))}$\\
Конкатенация строк
\end{itemize}
\subsection{Машина Тьюринга}
\label{sec-12-3}
Машина тьюринга состоит из ленты, головки, регистра состояния и конечной таблицы состояний.\\
Более формально, это 7-кортеж: $<Q, \Gamma, b, \sum, \sigma, q_0, F>$\\
Конечный список состояний, конечный алфавит, пустой символ из алфавита, символы, которые мы можем писать (из $\Gamma \backslash b$), функция таблицы состояний, начальное состояние из $Q$, конечное состояние из $Q$.
\begin{itemize}
\item Лента -- бесконечный двусвязный список, в каждой ячейке которого содержится символ из конечного алфавита, в котором также есть пустой символ (тут и далее), которым изначально заполнена вся лента
\item Головка может находиться над элементом, писать в него и читать из него символ. Может двигаться влево-вправо (или двигать ленту, неважно)
\item Регистр состояния хранит состояние -- элемент из конечного множества состояний машины. Есть особые состояния - стартовое и конечные.
\item Таблица состяний -- таблица, хранящая данные о функции смены состояния -- $foo: \Gamma \times Q \to \Gamma \times Q \times \{left, this, right\}$.\\
Функция берет текущее состояние, читает символ на головке, потом получает тройку, пишет новый символ, перемещается по третьему элементу, выставляет новое состояние. Если состояние конечное, то она останавливается.
\end{itemize}
Мы будем придерживаться нотации $<\_, \_,  , \_, \_, S, F>$.
\myworries{Что это за хуйня?}
\subsection{Проблема останова}
\label{sec-12-4}
Дано описание процедуры и входные данные. Функция $P(a, b)$ определяет, остановится ли $a$ на входных данных $b$. Существует ли $P$?
\begin{itemize}
\item Проблема останова неразрешима на машине Тьюринга:\\
Пусть $P$ существует.\\
Тогда $S(x) = P(x, x)$ остановится ли функция на своем же коде\\
$MyProg(x) = if S(x) then while(true)\{\} else 1$ \myworries{Пиздос тут код какой-то, надо бы нахехлатехать}
Рассмотрим $MyProg(`MyProg)$:\\
Если оно остановится, то первое условие выполнено, тогда оно не остановится и наоборот.\\
Значит, $P$ не существует.
\end{itemize}
\subsection{Выводимость и рек. функции - Тьюринг}
\label{sec-12-5}
\subsubsection{Выражение машин Тьюринга через рекурсивные функции}
\label{sec-12-5-1}
Мы хотим доказать, что если у нас есть какая-нибудь процедура, которую можно выразить в Тьюринге, то мы можем ее сделать и в формальной арифметике (рекурсивные функции представимы).\\
Введем обозначение $\langle st,tape,pos \rangle = 2^{st}*3^{tape}*5^{pos}$\\
Такая тройка -- основная характеристика машины в данный момент.\\
Будем называеть ее текущим полным состоянием, например.\\
$st$, $tape$, $pos$ -- геделевы нумералы, $st$ -- нумерал из $1$ элемента с состоянием, $tape$ - строка, обозначающая ленту (бесконечные слева и справа не входят), $pos$ - позиция в ленте.
\begin{itemize}
\item $p: \langle st, a \rangle \to \langle st, a, dir \rangle$\\
Принимает $\langle st, a \rangle$, декодит, лезет в $\sigma$ машины тьюринга, достает новые значения, делает из них $\langle ,,\rangle$ (\myworries{Шта}), отдает.
\item $t: \langle st \rangle \to 0 | 1$\\
Определяет, терминально ли наше состояние ($0$ если терминально)
\item $\epsilon$  -- пустой символ (у нас  )
\item $pb$, $pc$ кодируют $\beta$-функцией последовательность инпутов в последовательность аутпутов. $\beta(p_{b}, p_{c}, x) = p(x)$
\item $tb$, $tc$ аналогично кодируют $t$
\item $R\template{f, g}(\langle s_{st}, s_{tape}, s_{pos} \rangle, \epsilon , pb, pc, tb, tc, y)$
Запускает машину Тьюринга от стартового состояния, заранее говоря ей, сколько шагов (y) она должна сделать.\\
Возвращает тройку $\langle st, tape, pos \rangle$
\item Определим $f$, $g$
\begin{enumerate}
\item Дополнительные функции
\begin{itemize}
\item $os(prev) = \plog(prev, 1)$\\
Текущее состояние
\item $ot(prev) = \plog(prev, 2)$\\
Лента
\item $op(prev) = \plog(prev, 3)$\\
Позиция головки в ленте
\item $nextstate(pb, pc, prev) = \beta(pb, pc, 2^{os(prev)} * 3^{\plog(ot(prev), op(prev))}$\\
Реализует функцию p
\item $st(pb, pc, prev) = \plog(nextstate(pb, pc, prev), 1)$\\
Новое состояние.
\item $sym(pb, pc, prev) = \plog(nextstate(pb, pc, prev), 2)$\\
Символ который нужно писать
\item $dir(pb, pc, prev) = \plog(nextstate(pb, pc, prev), 3)$\\
Направление для перехода головки
\item $repl(pb, pc, prev) = (ot(prev) / (P_{op})^{\plog(ot(..), op(..))}) * (P_{op})^{sym}(..)$
Возвращает ленту, в которой удален символ в позиции op, и добавлен новый символ в эту же позицию.
\end{itemize}
\item $f$ -- возвращает полное состояние машины\\
$f(\langle start_{state} \rangle, \epsilon , pb, pc, tb, tc) = \langle start_{state} \rangle$
\item $g$ -- возвращает новое полное состояние из машины после перехода (пометка: $0$ - $nothing$, $1$ - $right$, $2$ - $left$ все фукнции вызываются с аргументом $prev, \langle start_{state} \rangle$ не используется)\\
$g(\langle start_{state} \rangle, \epsilon , pb, pc, tb, tc, y, prev) =$
\begin{center}
\begin{tabular}{lrl}
Condition & Result & Descr\\
\hline
$dir = 0$        & $\langle st, repl, op \rangle$ & nothing\\
$dir = 1 \& len(repl) = op$ & $\langle st, repl @ 2^{\epsilon}, op + 1 \rangle$ & tape end\\
$dir = 1$        & $\langle st, repl, op + 1 \rangle$ & move right\\
$dir = 2 \& op = 0$    & $\langle st, 2^{\epsilon} @ repl, op - 1$ & tape start\\
$dir = 2$      & $\langle st, repl, op - 1 \rangle$ & move left\\
\hline
\end{tabular}
\end{center}
\end{enumerate}
\item $steps$ -- функция, определяющая необходимое кол-во шагов

$steps(\langle \const{start\_state} \rangle, \epsilon , pb, pc, tb, tc) =$\\
$\mu \langle \beta(tb, tc, \plog(R\template{f, g}, 1))\rangle (\langle \const{start\_state} \rangle, \epsilon , pb, pc, tb, tc)$\\
Она найдет такое минимальное $k$, что состояние $\plog(R\template{f, g}(args, k), 1)$ терминально.
\end{itemize}
\subsubsection{Выражение программы по проверке доказательства в машине тьюринга}
\label{sec-12-5-2}
\begin{itemize}
\item $Emulate(input, prog) = \plog(R\template{f, g}(\langle \Godel{S}, input, 0 \rangle,  , pb, pc, tb, tc, steps(-//-)), 1) == F$\\
Функция проверяет, правда ли получившееся терминальное состояние - ок.\\
Можем давать программу такую, что она заканчивается в терминальном $F(inish)$ или в терминальном $FAIL$\\
Дает в качестве аргумента функцию перехода, $pb$, $pc$ выражают $prog$
\item
$\begin{multlined}[t]
Proof(term, proof) = Emulate(proof, MY\_PROOFCHECKER) \\
\&\& (\plog(proof, len(proof)) = term)
\end{multlined}$\\
Проверяет, что доказательство $p$ заканчивается корректно и его последний элемент -- то, что мы доказываем.
\item Любая представимая в ФА функция является рекурсивной\\
Пусть $f$ представима\\
Пусть $f(x_1, \dotsc, x_n) = b$, тогда $\vdash \phi(\overline{x_1}, \dotsc \overline{x_n}, \overline{b})$\\
Всегда можно построить рекурсивную функцию $G_{\phi(x_1, \dotsc, x_n, b, p)}$, утверждающую, что $p$ -- гёделев номер вывода предиката $\phi(\overline{x_1}, \dotsc, \overline{x_n}, \overline{b})$\\
Мы делаем это обычным перебором чисел, проверяем вывод нашей программой из домашнего задания, выраженную в тьюринге, а потом в
рекурсивных функциях.\\
Тогда $f$ в рекурсивных функциях выражается так:
$f(x_1\dots x_n) = \plog(\mu\template{S\template{G_{\phi}, U_{n+1}^1, \dotsc, U_{n+1}^n, $\\
$\plog(U_{n+1}^{n+1}, 1),$\\
$\plog(U_{n+1}^{n+1}, 2)}}(x_1, \dotsc, x_n), 1)$\\
Такая функция берет $\plog(1)$ от первого такого минимального геделева номера $k$ (геделева пара из двух элементов - $\langle b, p \rangle$)), что:
\begin{gather*}
S\template{G_{\phi}, U \dots}(x_1, \dotsc, x_n, k) = 0
\shortintertext{Это значит, что:}
G_{\phi}(x_1, \dotsc, x_n, \plog(k, 1), \plog(k, 2)) = 0 \\
\shortintertext{И значит, что:}
G_{\phi}(x_1, \dotsc, x_n, b, p) = 0
\shortintertext{То есть $p$ -- вывод:}
\phi(\overline{x_1}, \dotsc, \overline{x_n}, \overline{b}). \\
\shortintertext{Этот гёделев номер -- $b$}
\end{gather*}
\end{itemize}
%\section{Первая теорема о неполноте}
\label{sec-13}
\subsection{Непротиворечивость, \texorpdfstring{$\omega$}{ω}-непротиворечивость}
\label{sec-13-1}

\begin{definition}
    Теория \emph{непротиворечива}, если в ней нельзя вывести
    одновременно $a$ и $\lnot a$ (что аналогично невозможности
    вывести $a\land \lnot a)$.
\end{definition}

\begin{lemma}
    В противоречивой теории доказуема любая формула
\end{lemma}
\begin{proof}
    Противоречивость эквивалентна доказуемости $a \land \lnot a$. По \eqref{provability},
    \[a \to \lnot a \to b\]
    Кроме того, по \eqref{impltoand},
    \[(a \to \lnot a \to b) \to (a \land \lnot a \to b),\]
    из чего немедленно следует утверждение леммы.
\end{proof}

\begin{definition}
    Теория \emph{$\omega$-непротиворечива}, если из $\forall \phi(x) \vdash \phi(\overline{x})$ следует
    $\nvdash \exists p ~ \lnot \phi(p)$. Проще говоря, если мы взяли
    формулу, то невозможно вывести одновременно $\exists x ~ \lnot A(x)$
    и $A(0), A(1), \dotsc$
\end{definition}

\begin{lemma}[о $\omega$- и обычной непротиворечивости]
    Если теория $\omega$-непротиворечива, то она непротиворечива
\end{lemma}
\begin{proof}
    Рассмотрим следующую формулу:
\begin{gather*}
    \phi = (x=x \to x=x) \\
    \shortintertext{Такая формула очевидно доказуема $(A \to A)$}
    \vdash \phi[x:=k] k \in N_0 \\
    \shortintertext{Но недоказуемо $\exists x\lnot (x=x\to x=x)$}
\end{gather*}
Однако в противоречивой теории доказуемо всё. Значит, наша теория непротиворечива.
\end{proof}
\subsection{Первая теорема о неполноте}
\label{sec-13-2}
Определим отношение $W_1(x, p)$, истинное тогда и только тогда,
когда $x$ - геделев номер формулы $\phi$ с единственным свободным
аргументом $x$, а $p$ - геделев номер доказательства $\phi("\phi")$. Это
отношение выразимо в ФA, потому что мы просто пихаем это в наш
Proof, а его мы выразили через рекурсивные функции, а они
представимы.

Пусть его выражает $w_1(x, p)$;

Рассмотрим формулу $\sigma  = \forall p ~ \lnot w_1(x, p)$ -- для любого доказательства
оно не является доказательством самоприменения $\phi$, то есть
самоприменение $\phi$ недоказуемо.
То есть если $\sigma (\overline{\Godel{a}})$ истинно, то $a(\overline{\Godel{a}})$ недоказуемо.
В нашем случае если $\sigma (\overline{\Godel{a}})$ истинно, то $\sigma (\overline{\Godel{\sigma}})$ недоказуемо.
\begin{enumerate}
    \item Если формальная арифметика непротиворечива, то недоказуемо $\sigma (\overline{\Godel{\sigma}})$
    \begin{enumerate}
        \item Пусть $\vdash \sigma (\overline{\Godel{\sigma}})$, тогда найдется геделев номер ее док-ва p,
            тогда $W_1(\Godel{\sigma}, p)$, то есть $\vdash w_1(\overline{\Godel{\sigma}}, \overline{p})$.
        \item С другой стороны,
        \begin{gather*}
            \vdash \sigma (\overline{\Godel{\sigma}})\\
            \vdash \forall p ~ \lnot w_1(\overline{\Godel{\sigma}}, p)\\
            \forall p ~ \lnot w_1(\overline{\Godel{\sigma}}, p) \to \lnot w_1(\overline{\Godel{\sigma}}, \overline{p})\\
            \lnot w_1(\overline{\Godel{\sigma}}, \overline{p})
        \end{gather*}
        Тогда ФА противоречива.
    \end{enumerate}
    \item Если формальная арифметика $\omega$-непротиворечива, то недоказуемо $\lnot \sigma (\overline{\Godel{\sigma}})$
    Пусть $\vdash \lnot \sigma (\overline{\Godel{\sigma}})$,
    то есть $\vdash \lnot \forall p\lnot w_1(\overline{\Godel{\sigma}}, p)$, что значит
    $\exists p.w_1(\overline{\Godel{\sigma}}, p)$
    Найдется такой $q$, что $\vdash w_1(\overline{\Godel{\sigma}}, \overline{q})$, потому что если бы не нашелся,
    это бы значило доказуемость для каждого $q \lnot w_1(\overline{\Godel{\sigma}}, \overline{q})$, тогда по
    $\omega$-непротиворечивости было бы не доказуемо $\exists p\lnot \lnot w_1(\overline{\Godel{\sigma}}, p)$
    То $q$, что мы нашли -- это номер доказательства  $\sigma (\overline{\Godel{\sigma}})$, что и
    утверждает выражение $\vdash w_1(\overline{\Godel{\sigma}}, \overline{q})$. Но мы предполагали, что $\vdash \lnot \sigma (\overline{\Godel{\sigma}})$.
    Противоречие.
\end{enumerate}

\subsubsection*{Нормальное доказательство общезначимости}

Я не знаю, зачем нам второй пункт, но из первого следует, что если
наша теория $\omega$-непротиворечива, то она непротиворечива (по лемме выше),
значит в ней недоказуемо $\sigma (\overline{\Godel{\sigma}})$, то есть $\forall p ~ \lnot w_1(\overline{\Godel{\sigma}}, p)$, то есть
по корректности последнее выражение И, но это и есть в точности определение
$\sigma (\overline{\Godel{\sigma}})$.

\subsubsection*{Ненормальное д-во общезначимости}

Итого мы доказали, что если формальная арифметика $\omega$-непротиворечива,
то в ней не доказуемо ни $\sigma (\overline{\Godel{\sigma}})$ ни $\lnot \sigma (\overline{\Godel{\sigma}})$. Одно из них точно тавтология
(в формуле нет свободных переменных). Тогда ФА неполна при условии
$\omega$-непротиворечивости.

\subsubsection*{Другое доказательство общезначимости}

$\lnot \sigma (\overline{\Godel{\sigma}})$ недоказуема
\[
    \llbracket \sigma (\overline{\Godel{\sigma}}) \rrbracket = \llbracket \forall p\lnot w_1(\overline{\Godel{\sigma}}, p) \rrbracket =
    \begin{cases}
    \true & \text{если $\llbracket \lnot w_1(\overline{\Godel{\sigma}}, a)\rrbracket = \true$ для какого-то $a$} \\
    \false & \text{иначе}
    \end{cases}
\]

Это значит, что\\
И если $\llbracket w_1(\overline{\Godel{\sigma}}, a) \rrbracket = \false$\\
$\llbracket w_1(\overline{\Godel{\sigma}}, a) \rrbracket$ = \false, докажем от противного
Пусть $\llbracket \sigma (\overline{\Godel{\sigma}}) \rrbracket = \false$,\\
$\llbracket \forall p\lnot w_1(\overline{\Godel{\sigma}}, p) \rrbracket = \false$\\
$\llbracket \lnot \forall p\lnot w_1(\overline{\Godel{\sigma}}, p) \rrbracket = \true$\\
$\llbracket \exists p.w_1(\overline{\Godel{\sigma}}, p) \rrbracket = \true$\\
$\llbracket w_1(\overline{\Godel{\sigma}}, a) \rrbracket = \true$ для какого-то а\\
то есть a доказывает $\sigma (\overline{\Godel{\sigma}})$
\myworries{Какая-то шляпа, если честно}

???

тогда по определению $w_1$ существует
доказательство $\sigma (\overline{\Godel{\sigma}})$,
\subsection{Пример \texorpdfstring{$\omega$}{ω}-противоречивой, но непротиворечивой теории (при усл. непрот. ФА)}
\label{sec-13-3}
Добавим в ФА аксиому Г: $\lnot \sigma (\overline{\Godel{\sigma}})$
Тогда по контрпозиции 1п2 она $\omega$-противоречива.
Если бы мы могли доказать противоречивость нашей системы, то
ФА была бы противоречива, тогда хз
$\lnot \sigma (\overline{\Godel{\sigma}}) \vdash \sigma (\overline{\Godel{\sigma}})\land \lnot \sigma (\overline{\Godel{\sigma}})$
$\vdash \sigma (\overline{\Godel{\sigma}})$
Но мы предположили что $\lnot \sigma (\overline{\Godel{\sigma}})$
\subsection{Форма Россера}
\label{sec-13-4}
Если формальная арифметика непротиворечива, то в ней найдется
такая формула $\phi$, что $\nvdash \phi$ и $\nvdash \lnot \phi$

\section{Ticket 12: 2-ая теорема Гёделя о неполноте}
\label{sec-14}
\subsection{Consis, Условия выводимости Гильберта-Бернайса}
\label{sec-14-1}
Определим $Consis$ как утверждение, показывающее непротиворечивость ФА - отсутствие $\phi$: $\vdash \phi, \lnot \phi$. Поскольку
в противоречивой теории выводится что угодно, возьмем что-то недоказуемое, например $1 = 0$.\\
$Consis = \forall p(\lnot Proof(\overline{\Godel{1 = 0}}, p))$

Определим отношение $Sub(a, b, c)$ истинным, если $a$, $b - \Godel{a}$, $\Godel{b} \land c = \Godel{a[x := b]}$ или же $a \lor b$ не геделев номер и $c = 0$

Пусть $Sub(a, b, c)$ выражает $\tau(a, b, c)$
\begin{lemma}
Лемма о самоприменении
$a(x)$ - формула, тогда $\exists b$ такой что
\begin{enumerate}
\item $\vdash a(\overline{\Godel{b}}) \to b$
\item $\vdash \beta \to a(\overline{\Godel{b}})$
\end{enumerate}
\end{lemma}
\begin{proof}
$b_0(x) = \forall t(\tau(x, x, t) \to a(t))$\\
$b = b_0(\overline{\Godel{b_0}})$
\begin{enumerate}
\item $a(\overline{\Godel{b}}) \vdash a(\overline{\Godel{b}})$\\
$a(\overline{\Godel{b}}) \vdash \tau(\overline{\Godel{b_0}}, \overline{\Godel{b_0}}, \overline{\Godel{b}}) \to a(\overline{\Godel{b}})$    акс 1 + MP\\
$a(\overline{\Godel{b}}) \vdash \top \to (\tau(\overline{\Godel{b_0}}, \overline{\Godel{b_0}}, \overline{\Godel{b}}) \to a(\overline{\Godel{b}}))$\\
$a(\overline{\Godel{b}}) \vdash \top \to \forall t(\tau(\overline{\Godel{b}}, \overline{\Godel{b_0}}, t) \to a(t))$\\
$a(\overline{\Godel{b}}) \vdash \forall t(\tau(\overline{\Godel{b_0}}, \overline{\Godel{b_0}}, t) \to a(t))$\\
$a(\overline{\Godel{b}}) \vdash b$
\item $b \vdash \forall t(\tau(\overline{\Godel{b_0}}, \overline{\Godel{b_0}}, t) \to a(t))$    тут почти $a \vdash a$ написано\\
$b \vdash \tau(\overline{\Godel{b_0}}, \overline{\Godel{b_0}}, \overline{\Godel{b}})$             по выразимости\\
$b \vdash \tau(\overline{\Godel{b_0}}, \overline{\Godel{b_0}}, \overline{\Godel{b}}) \to a(\overline{\Godel{b}})$    сняли квантор с 1\\
$b \to a(\overline{\Godel{b}})$
\end{enumerate}
\end{proof}
\begin{lemma}
Условия Гильберта-Бернайса \myworries{Лемма, теорема? хз чёт кароч}\\
Пусть $\pi g(x, p)$ выражает $Proof(x, p)$\\
$\pi r(x) = \exists t \pi g(x, t)$ тогда если
\begin{enumerate}
\item $\vdash a$, то $\vdash \pi r(\overline{\Godel{a}})$
\item $\vdash \pi r(\overline{\Godel{a}}) \to \pi r(\overline{\Godel{\pi r(\overline{\Godel{a}})}})$
\item $\vdash \pi r(\overline{\Godel{a}}) \to \pi r(\overline{\Godel{a \to b}}) \to \pi r(\overline{\Godel{b}})$
\end{enumerate}
\end{lemma}
\subsection{Вторая теорема о неполноте}
\label{sec-14-2}
\begin{theorem}
$Consis$ не доказуем
\end{theorem}
\subsubsection{Рукомашеское доказательство без условий Гильберта-Бернайса}
\label{sec-14-2-1}
\begin{proof}
Докажем несколько различных фактов
\begin{itemize}
\item Если арифметика непротиворечива, в ней нет доказательства $Consis$:\\
Рассмотрим $Consis \to \sigma(\overline{\Godel{\sigma}})$. Тогда если $Consis$ доказуемо, то $\sigma(\overline{\Godel{\sigma}})$ недоказуемо. Тоесть это формулировка 1-ой теоремы Гёделя о неполноте.\\
Тогда если у нас будет $Consis$, мы сможем доказать $\sigma(\overline{\Godel{\sigma}})$, тогда 1-ая теорема Гёделя о неполноте не работает. Значит $Consis$ недоказуемо.
\item Доказательство того, что $Consi$s недостаточно формален:\\
Заменим $Consis$ в д-ве на:\\
$Proof1(a, x) = Proof(a, x) \land \lnot Proof(\Godel{1=0}, x)$\\
$Consis1 = \forall x \lnot Proof1(\Godel{1=0}, x)$\\
Если арифметика непротиворечива, то $Proof1(a, x) = Proof(a, x)$\\
Если арифметика противоречива, то $Consis1$ доказуема как и все остальное.
Ну давайте менять.

Поменяли. Смотрим. хехехе, давайте докажем $Consis1$:\\
$\lnot (\pi(x) \& \lnot \pi(x))$ (Доказуемо в ИВ)\\
$\top$\\
$\top \to \lnot (\pi(x) \& \lnot \pi(x))$ (1 акс, MP)\\
$\lnot (\pi(x) \& \lnot \pi(x))$\\
$\forall x(\lnot (\pi(x) \& \lnot \pi(x)))$

Тогда выходит, что мы можем доказать противоречивость арифметики. Но это не так, бага вот в чем:
Замена $Consis$ на $Consis1$ неоправдана -- в $Consis1$ есть формула $1=0$, на которой ее результат не вычисляется, а
постулируется. Чтобы выражать $Consis$ абстрактно, существуют условия выводимости Гильберта-Бернайса.

Докажем, что $Consis1$ не удовлетворяет 3-ему свойству Г-Б:\\
Пусть $Proof1(x,p)$ выражает $\pi 1$.
$\vdash \pi 1(\overline{\Godel{a}}) \to \pi 1(\overline{\Godel{a \to b}}) \to \pi 1(\overline{\Godel{b}})$ оценим при $a=(2=0)$,$b=(1=0)$\\
$? \to (true \to false)$\\
$? \to false$

Если эта формула верна, то $\vdash \pi 1(\overline{\Godel{a}})$
Тогда если $\pi 1(\overline{\Godel{a}})$, то $Proof(2=0, x) \land \lnot Proof(`1=0, x) = И$\\
Это значит что теория противоречива, потому что в ней выводимо $2=0$, но она непротиворечива, потому что недоказуемо $1=0$.
\end{itemize}
\end{proof}
\subsubsection{Доказательство 2 теоремы Гёделя о неполноте}
\label{sec-14-2-2}
\begin{proof}
Пусть $\pi$ удовлетворяет условиям Гильебрта-Бернайса:\\
$Consis = \lnot \pi(1=0)$\\
ФА непротиворечива\\
Тогда $\not \vdash Consis$

\begin{enumerate}
\item По лемме о самоприменении
\begin{enumerate}
\item $\lnot \pi(\gamma) \to \gamma$
\item $\gamma \to \lnot \pi(\gamma)$
\item $\lnot \gamma \to \pi(\gamma)$ (контрпозиция)
\item $\pi(\gamma) \to \lnot \gamma$
\end{enumerate}
\item $\pi(\gamma) \to \pi(\lnot \gamma)$
\begin{enumerate}
\item $\pi(\gamma) \vdash \pi(\overline{\Godel{\pi(\gamma)}})$ (ГБ 2)
\item $\vdash \pi(\pi(\gamma) \to \lnot \gamma)$ (ГБ 1 от 1.4)
\item $\vdash \pi(\pi(\gamma)) \to \pi(\pi(\gamma) \to \lnot \gamma) \to \pi(\lnot (\gamma))$ (ГБ 3)
\item $\pi(\gamma) \vdash \pi(\lnot \gamma)$  (2 MP (2.1, 2.2))
\end{enumerate}
\item $\vdash \pi(\alpha \to \beta \to \gamma) and \vdash \pi(\alpha) \to \pi(\beta) => \vdash \pi(\alpha) \to \pi(\gamma)$
\begin{enumerate}
\item $\pi(\alpha \to \beta \to \gamma) \to \pi(\alpha) \to \pi(b \to \gamma)$ (ГБ 3)
\item $\pi(\beta \to \gamma) \to \pi(\beta) \to \pi(\gamma)$ (ГБ 3)
\item $\pi(\alpha) \to \pi(\beta \to \gamma)$ (MP 1, given)
\item $\pi(\alpha) \to \pi(\beta)$ (given)
\item $\pi(\alpha) \to \pi(\gamma)$ (занести под дедукцию, ГБ 3)
\end{enumerate}
\item $\vdash \pi(\gamma) \to \pi(1=0)$
\begin{enumerate}
\item $\gamma \to \lnot \gamma \to (1=0)$ (10i в ИИВ, выводима в предикатах)
\item $\vdash \pi(\gamma \to \lnot \gamma \to (1=0))$ (ГБ1)
\item $\pi(\gamma) \to \pi(\lnot \gamma)$ 2
\item $\vdash \pi(\gamma) \to \pi(1=0)$ (MP 4.2, 4.3)
\end{enumerate}
\item $\not \vdash Consis$
$\vdash \lnot \pi(1=0) \to \lnot \pi(\gamma)$ (контрапозиция 4)\\
$\vdash Consis \to \lnot \pi(\gamma)$ (the same)\\
$] \vdash Consis$, тогда $\vdash \lnot \pi(\gamma)$\\
$\vdash \lnot \pi(\gamma) \to \gamma \Rightarrow \vdash \gamma \Rightarrow \vdash \pi(\gamma)$ (1.1, ГБ1)\\
$\vdash \lnot \pi(\gamma), \vdash \pi(\gamma)$
\end{enumerate}
\end{proof}

%\section{Ticket 13: ТМ}
\label{sec-15}
\subsection{Теория множеств}
\label{sec-15-1}
   Значит это такая теория первого порядка.
   В сигнатуре модели есть один пред.символ - \in
   Добавляем связку a \leftrightarrow b = (a \to b) \& (b \to a)
   σ \in Θ => \forall x(x \in σ \to x \in Θ)
o   σ = Θ => σ \in Θ \& Θ \in σ
   ∅ : \forall x(\lnot x \in ∅)
   x ∩ y = z, тогда \forall t(t \in z \leftrightarrow t \in x \& t \in y)
   Dj(x) \forall a\forall b(a \in x \& b \in x \& a \ne  b \to a ∩ b = ∅)
   X(a) - мн-во всех x пересекающихся ровно в одном эл-те с каждым из а
   и содержащих элементы из ∪a.
   X(\{\{1, 2\}, \{2', 3\}\}) = \{\{2, 3\}, \{1, 2'\}\}
\subsubsection{Если существует мн-во, то существует пустое мн-во}
\label{sec-15-1-1}
Аксиома выделения:
\forall x\exists b\forall y(y \in b \leftrightarrow (y \in x \& \phi(y)))
Возьмем наше существующее мн-во x
\exists b\forall y(y \in b \leftrightarrow (y \in x \& \phi(y)))
Пусть \phi(y) = \bot
тогда подставим ∅ вместо b:
\forall y(y \in ∅ \leftrightarrow (y \in x \& \bot))
Это выполняется вроде.
\subsubsection{Если x, то найдется \{x\}}
\label{sec-15-1-2}
\forall x\exists \{x\}\forall y(y \in \{x\} \to y = x)
\begin{enumerate}
\item Пусть x \ne  ∅
\{x\} = \{y | y \in \{x, y\} \& y \ne  ∅\}
по аксиоме объединения \exists p\forall y(y \in p \leftrightarrow \exists s(y \in s \& s \in x))
\forall y(y \in \{x, ∅\} \leftrightarrow \exists s(y \in s \& s \in x))

или по аксиоме пары
\exists p(x \in p \& ∅ \in p \& \forall z(z \in p \to (x = z \lor y = z)))
x \in \{x, ∅\} \& ∅ \in \{x, ∅\} \& \forall z(z \in \{x, ∅\} \to \dots \}

ДГ руками помахал тут, ну и я помахаю по причине
отсутствия времени доказывать

А, нет, вот, кажется:
По аксиоме степени \forall x\exists \{x, ∅\}\forall y(y \in \{x, ∅\} \leftrightarrow y \in x)
\forall x\exists \{x, ∅\}\forall y((y \in \{x, ∅\} \to y \in x)\&(y \in x \to y \in \{x, ∅\}))
\lnot y \in ∅, значит (y \in x \to y \in x) = \top
\forall x\exists \{x, ∅\}\forall y((y \in \{x, ∅\} \to y \in x)\&\top)
\forall x\exists \{x, ∅\}\forall y(y \in \{x, ∅\} \to y = x)   более слабое условие
\item x = ∅
P(∅) = \{∅\}
\end{enumerate}
\subsubsection{\exists !x(\forall y.\lnot (y \in x))}
\label{sec-15-1-3}
\exists x(\forall y.\lnot (y \in x)) \& \forall a\forall b((\forall y.\lnot (y \in a)) \& (\forall y.\lnot (y \in b)) \to a = b)
Первое по определению пустого множества и аксиоме выделения с \bot
\forall y.(\lnot y \in \{\}) \& \forall y.(\lnot y \in \{\}) \to \forall p((p\in x \to p\in y) \& (p\in y \to p\in x))
Второе как-то через ∅_1 \in ∅₂ и обратное включение
На основании того, что мы подставляем наши пустые множества, импликация
вырождается в \top \to \top
\subsubsection{x ∩ y существует}
\label{sec-15-1-4}
по теореме выделения
\forall x\exists b\forall y(y \in b \leftrightarrow (y \in x \& \phi(y)))
\forall y(y \in x ∩ y \leftrightarrow (y \in x \& t \in y))
\subsection{Аксиоматика ZFC}
\label{sec-15-2}
\subsubsection{Аксиома равенства}
\label{sec-15-2-1}
\forall x\forall y\forall z((x = y \& y \in z) \to x \in z)
Eсли два множества равны, то любой элемент лежащий в первом,
лежит и во втором
\subsubsection{Аксиома пары}
\label{sec-15-2-2}
\forall x\forall y(\lnot (x=y) \to \exists p(x \in p \& y \in p \& \forall z(z \in p \to (x = z \lor y = z))))
x \ne  y, тогда сущ. \{x, y\}
\subsubsection{Аксиома объединений}
\label{sec-15-2-3}
\forall x(\exists y(y\in x) \to \exists p\forall y(y \in p \leftrightarrow \exists s(y \in s \& s \in x)))
Eсли x не пусто, то из любого семейства множеств можно
образовать „кучу-малу“, то есть такое множество p,
каждый элемент y которого принадлежит по меньшей мере
одному множеству s данного семейства s x
\subsubsection{Аксиома степени}
\label{sec-15-2-4}
\forall x\exists p\forall y(y \in p \leftrightarrow y \in x)
P(x) - множество степени x (не путать с 2ˣ - булеаном)
Это типа мы взяли наш x, и из его элементов объединением и
пересечением например понаобразовывали кучу множеств, а потом
положили их в p.
\subsubsection{Схема аксиом выделения}
\label{sec-15-2-5}
\forall x\exists b\forall y(y \in b \leftrightarrow (y \in x \& \phi(y)))
Для нашего множества x мы можем подобрать множество побольше,
на котором для всех элементов, являющихся подмножеством x
выполняется предикат.
\subsubsection{Аксиома выбора (не входит в ZF по дефолту)}
\label{sec-15-2-6}
Если a = Dj(x) и a \ne  0, то x \in a \ne  0
\subsubsection{Аксиома бесконечности}
\label{sec-15-2-7}
\exists N(∅ \in N \& \forall x(x \in N \to x ∪ \{x\} \in N))
\subsubsection{Аксиома фундирования}
\label{sec-15-2-8}
\forall x(x = ∅ \lor \exists y(y \in x \& y ∩ x = ∅))
\forall x(x \ne  ∅ \to \exists y(y \in x \& y ∩ x = ∅))
Равноценные формулы.

Я бы сказал, что это звучит как-то типа
"не существует бесконечно вложенных множеств"
\subsubsection{Схема аксиом подстановки}
\label{sec-15-2-9}
\forall x\exists !y.\phi(x,y) \to \forall a\exists b\forall c(c \in b \leftrightarrow (\exists d.(d \in a \& \phi(d, c))))
Пусть формула \phi такова, что для при любом x найдется единственный y
такой, чтобы она была истинна на x, y, тогда для любого a
найдется множество b, каждому элементу которого c можно сопоставить
подмножество a и наша функция будет верна на нем и на c
Типа для хороших функций мы можем найти множество с отображением из
его элементов в подмножество нашего по предикату.


%\section{Ticket 14: oрдиналы}
\label{sec-16}
\subsection{Ординальные числа}
\label{sec-16-1}
\begin{itemize}
\item Определение вполне упорядоченного множества (фундированное
с линейныи порядком).
\item Определение транзитивного множества:\\
Множество $X$ транзитивно, если
$\forall a \forall b((a \in b \land b \in x) \to a \in x)$
\item Ординал -- транзитивное вполне упорядоченное отношением $\in$ мн-во
\item Верхняя грань множества ординалов $S$:\\
$C | \{C = min(X) \land C \in X | X = \{z | \forall (y \in S)(z \geq y)\}\}$
$C = Upb(S)$
$Upb(\{\emptyset\}) = \{\emptyset\}$
\item $Successor ordinal$ (сакцессорный ординал?):\\
Это $b = a' = a \cup \{a\}$
\item Предельный ординал:\\
Ординал, не являющийся ни $0$ ни $successor$'ом.
\item Недостижимый ординал:\\
$\epsilon$ - такой ординал, что $\epsilon = w^{\epsilon}$\\
$\epsilon_0$ = $Upb(w, w^{w}$, $w^{w^{w}}$, $w^{w^{w^w}}$, \dots) -- минимальный из $\epsilon$ ординалов
\item Канторова форма -- форма вида $\sum(a*w^b+c)$, где $b$ -- ординал, последовательность строго убывает по $b$. Есть слабая канторова форма, где вместо $a$ ($a \in N$) пишут $a$ раз $w^b$. В канторовой форме приятно заниматься сложениями и прочим, потому что всякие $upb$ слишком ниочем.
\end{itemize}
\subsection{Операции над ординальными числами}
\label{sec-16-2}
\subsubsection{Стабилизация убывающей последовательности}
\label{sec-16-2-1}
Допустим, что есть убывающая последовательность ординалов $x_1, x_2, \dotsc$ Возьмем ординал $x_1 + 1 = x_0$. Тогда $\{x_1, x_2, \dots \} \in x_0$. $x_0$ не пусто, значит там есть минимальный элемент по определению порядка на ординале. Пусть этот элемент -- $m$. Тогда поскольку $m \in x_0$, то $m = x_i$ для какого-то $i$ нашей убывающей последовательности.
\begin{enumerate}
\item Последовательность убывает нестрого.\\
Тогда все $x_k \leq m$, для $k > i$. Это выполняется, если $x_k = x_i$, тогда последовательность стабилизируется в $m$.
\item Последовательность убывает строго.\\
Тогда все $x_k < m$ для $k > i$, но $m$ - минимум множества. Противоречие.\\
Убывающей строго последовательности ординалов не существует.
\end{enumerate}
\subsubsection{Арифметические операции через Upb}
\label{sec-16-2-2}
Пусть lim(a) = предельный ординал a
\begin{align*}
&x + 0 &= &x \\
&x + c' &= &(x + c)' \\
&x + lim(a) &= &Upb\{x + c | c < a\} \\
\\
&x * 0 &= &0 \\
&x * c' &= &x * c + x \\
&x * lim(a) &= &Upb\{x * c | c < a\} \\
\\
&x^{0} &= &1\\
&x^{c'} &= &x^c * x\\
&x^{lim(a)} &= &Upb\{x^c | c < a\}
\end{align*}

Ну вот короче можно так, только приходится много думать как реализовывать $Upb$. Или только у меня так.

$2^{w} = Upb(2, 4, 8, \dots ) = w$
\subsubsection{Арифметические операции через Канторову форму}
\label{sec-16-2-3}
Хорошо описано в этой статье:\\
\href{http://www.google.ru/url?sa=t&rct=j&q=&esrc=s&source=web&cd=1&ved=0CB4QFjAA&url=http://www.ccs.neu.edu/home/pete/pub/cade-algorithms-ordinal-arithmetic.pdf&ei=FDW6\lor JOYNuvXyQPd0ILQBQ&usg=AFQjCNENBOBOdKbbqBYN3iFhmAu_jFD2Sw&sig2=1UISFzJ_21I8f1YScX7Tkw&bvm=bv.83829542,d.bGQ&cad=rjt}{Algorithms for Ordinal Arithmetic}\\
Переписывать довольно громоздко, учитывая количество вспомогательных функций. Есть в гитхабе Михаила Волхова (\href{https://github.com/volhovM/mathlogic}{github}) реализованное


%\section{Ticket 15: кардиналы}
\label{sec-17}
\subsection{Кардинальные числа}
\label{sec-17-1}
Будем называть множества равномощными, если найдется биекция.
Будем называть A не превышающим по мощности B, если найдется
инъекция A \to B (|A| \le |B|)
Будем называть А меньше по мощности, чем B, если |A| \le |B| \& |A| \ne  |B|
Кардинальное число - число, оценивающее мощность множества.
Кардинальное число ℵ - это ординальное число a, такое что
\forall  x \le a |x| \le |a|
ℵ₀ = w по определению; ℵ_1 = минимальный кардинал, следующий за ℵ₀
Кардинальное число ℶ - это ординальное число а, такое что
ℶ_i = P(ℶ_i₋_1)
ℶ₀ = ℵ₀
Континуум-гипотеза формулируется таким образом: |P(ℵ₀)| = ℵ_1 или ℶ_1 = ℵ_1
В 40 году Гёдель доказал недоказуемость отрицания Континуум-гипотезы
в терминах ZFC, в 60 Коэн сделал то же самое но без отрицания. Это все
в условиях непротиворечивости ФА. То есть в ZFC нельзя доказать или
опровергнуть континуум-гипотезу.
Сложение кардинальных чисел - |A| + |B| = |A∪B| если в них
нету общих элементов, иначе max(|A|, |B|), поскольку мы можем
построить двумерную таблицу из перес. элементов.
Остальное есть на вики и вряд ли нужно вообще.
\subsection{Диагональный метод Кантора}
\label{sec-17-2}
Докажем, что для любого множества |x| < |P(x)|
Воспользуемся диагональным методом Kантора
Пусть |x|=|P(x)|
Выпишем таблицу, в которой  столбцу p и строке q соответствует
1, если в множестве X лежит p, а в множестве P(X) лежит
множество, содержащее в себе p. Построим ключевое мн-во t:
элемент лежит в t, если на i-й диагональной позиции не стоит 1
и наоборот. То есть это множество всех таких элементов из X,
которым по биекции соответствует множество о чем угодно, но не
о самом элементе (не включающее элемент).
t состоит из подмножеств X, тогда оно должно лежать в P(X).
Докажем, что строка t не присутствует в таблице, сравнив ее с
каждой другой строкой - от каждой n-й строки отличается в n-й
столбце по построению.
Противоречие - t нет в таблице, но t \in P(X).
\subsection{Теорема Лёвингейма-Скулема}
\label{sec-17-3}
\begin{itemize}
\item Назовем мощностью модели мощность ее носителя (\lor или P или \lor ∪P).
M - модель, |M| - ее мощность, ну ясно.
\item Элементарная подмодель
Пусть M - модель фс первого порядка с носителем D. Пусть определено
D_1 ⊂ D, тогда структура M_1 построенная на D_1 так, что в ее интерпретации
лежит все, что и в интерпретации M, кроме элементов, взаимодействующих
с M $\backslash$ M_1 (сужение области определения на D_1), называется \textbf{элементарной}
\textbf{подмоделью}, если:
\begin{enumerate}
\item Любая функция ФС, над которой рассматривается M, замкнута на
D_1 (то есть если a \in D_1, b \in D_1, \dots  то f(a, b, \dots ) \in D_1)
\item Любая формула A(x_1\dots x_n) теории при любых аргументах из D_1,
истинная в M истинна и в M_1.
\end{enumerate}
\item Элементарная подмодель теории - модель теории
Рассмотрим формулу А, она общазначима в М, значит и в М_1, тогда M_1 корректна.
\item Счетно-аксиоматизируемая теория - множество аксиом ФС имеет мощность ℵ₀
\item ФА и ТМ счетно-аксоиматизируемые
\item Пусть M - модель, T - мн-во формул теории. Тогда \exists M_1 : |M_1| = max(|T|, ℵ₀)
Нужно построить необходимое предметное множество и доказать,
что модель на нем - это подмодель.
\begin{enumerate}
\item Построение множества
Пусть у нас есть множество D', тогда D'' = D' ∪ P, где P - некоторое
множество формул добавленное при рассмотрении формул D' по одной.
A(y, x_1\dots x_n) - n-местная формула из Т. Фиксируем x_1\dots x_n из D'.
\begin{itemize}
\item Если А = И или А = Л (тождественно) при любом y \in D - пропустим формулу
\item Если А = И или А = Л при каких-то y \in D' - пропустим формулу
\item \exists y: A(y,..) = И, но при этом \forall y\in D' A(y,..) = Л - тогда добавим один из
тех у, на которых формула истинна, в D''. Добавим еще констант, которые
нужны для вычисления А. Типа если В D' не хватает переменных для того,
чтобы показать что A может принимать истинностное значение, сгенерим
и добавим такое.
\end{itemize}
Переход от предыдущего множества к текущему увеличивает его не более чем на
ℵ₀ * |T| * |D'| - max(ℵ₀, T)
Рассмотрим D₀, D₀ ⊂ D такое, что в него входят те элементы носителя,
соответствующие константам, упоминающимся в Т. Если оно пустое -- добавим
какую-нибудь константу из D. Оно ляжет в начало счетной последовательности
D₀ ⊂ D_1 ⊂ \dots  (каждый переход описан выше). D* = ∪D_i.
D* - нужное нам множество. |D*| = max(ℵ₀, T)
\item Проверка структуры
Индукция по структуре.
\begin{itemize}
\item База. Предикат.
P(f_1(x_1\dots x_n), \dots , fₖ(x_1\dots x_n)). Если x_1\dots x_n взяты из D*, то они были
добавлены на некотором шаге, значит \exists t | x_i \in Dₜ. Тогда на шаге Dₜ₊_1
лежат результаты функций f_1\dots fₖ. по построению. Тогда оценка формулы
сохраняется.
\item Переход
Связки X\&Y, X\lor Y, X\to Y, \lnot X работают на сужении модели и оценка сохр.
\begin{itemize}
\item \exists yB(y, x_1\dots x_n). Фиксируем x_1\dots x_n из D*.
\begin{enumerate}
\item A была тождественно истинна или ложна - все ОК
\item А принимала значения разных знаков
Каждый x_i добавлен на каком-то шаге, тогда возьмем максимальный
шаг t, в Dₜ₊_1 уже лежат все эти x_i.
Тогда по построению Dₜ₊_1 мы добавили нужный y такой что B(y, x_1\dots x_n)
определено и выполнено в M.
Значит B выполнено в M* по индукции, тогда A истинна в M*.
\end{enumerate}
\end{itemize}
\item \forall yB(y, x_1\dots x_n).
\begin{enumerate}
\item Тождественно - ОК
\item Принимает значения разных знаков
Если оно истинно в M, тогда оно истинно в M* по 1 пункту перехода.
Если \forall yB(y, x_1\dots x_n) было ложно на t-шаге, тогда на t+1 шаге мы
здоровски исправили ситуацию, положив в мир y на котором оно истинно.
Если оно было истинно, то по пункту 2 пошло дальше.
\end{enumerate}
Таким образом, D* - подмодель нашего множества, |D*| = ℵ₀ + |T|
\end{itemize}
\end{enumerate}
\end{itemize}
\subsection{Парадокс Скулема}
\label{sec-17-4}
Мнимый парадокс Сколема формулируется так:
Возьмем теорию, прикрутим модель с аксиоматикой ZF. Модель
будет счетно-аксиоматизируемой потому что ZF.
Утверждается, что в ZF $\vdash \exists x(|x| = ℶ_1)$ - это доказывает диагон.
метод Кантора.
Тогда получается что по теореме Лёвингейма-Скулема у нашей
модели есть подмодель размером ℵ₀ + ℵ₀(счетно-акс) = ℵ₀, но
мы можем взять то самое x | |x| = ℶ_1, и его занумеровать, выходит.

Формальный подход не допускает этого конфликта ввиду одного
простого факта:
Рассмотрим отношение существования несчетного мн-ва в R.
$ZF \vdash \lnot \exists f$(f - биецкия между w и P(w)) \& \exists f(f - биецкия между w и w∪w)
// второе гарантирует счетность
Первый аргумент конъюнкции - \lnot \exists f(\forall x\forall y(<x,y>\in f \leftrightarrow x\in A\&x\in B))
<x,y> - пара (типа \{x, y, \{x\}\})
Тогда это значит, что в носителе модели нет такого f, что он
бы представлял собой объединение пар.
Собственно, по теореме Лёвингейма-Скулема у нас любая подмодель
будет иметь счетный носитель. Нет никакого противоречия, потому
что мы все еще работаем со счетным количеством множеств, а
отсутствие биекции все так же выражается отсутствием множества
в носителе.


%\section{Ticket 16: неполнота ФА}
\label{sec-18}
\subsection{Теорема о трансфинитной индукции}
\label{sec-18-1}
Пусть есть формула с одной свободной переменной a(x)
a истинна, если
\begin{enumerate}
\item a(0)
\item Если для любого конечного p - ординала мы можем
показать следование \{ q < p => a(p) \}, то a(p) истинно.
\end{enumerate}

Без док-ва, не требуется.

\subsection{Построение S∞}
\label{sec-18-2}
Мы строим еще одну теорию I порядка.
По сути, мы вкладываем ФА в нашу теорию так, что
любое доказательство ФА работает в S∞ и мы можем доказать
доказать непротиворечивость любого "импортированного" д-ва

\begin{enumerate}
\item Формулы:
Оставим связки \forall x, \lor , \lnot
Заметим, что \{\lor , \lnot \} полно для \{0, 1\}.
\item Доказательство
Доказательством является дерево утверждений, в
узлах которого правли, причем если дерево
растет вверх, то правила действуют сверху вниз.
\item Аксиомы:
\begin{enumerate}
\item все термы ФА без переменных типа θ_1=θ₂ (корректные)
\item все термы вида \lnot (θ_1=θ₂) если [θ_1]\ne [θ₂] (некорректные, все остальн.)
\end{enumerate}
\item Правила:
Примечание: в правилах используются боковые формулы,
они могут отсутствовать. Это сделано для формализации
того факта, что мы можем применять правило для двух любых
элементов нашей дизъюнкции или вроде того.
Примечение: org-mode подчеркивает a, если "$_{\text{a}}$\_"
\begin{enumerate}
\item Структурные
\begin{enumerate}
\item Перестановка
\uline{a\lor b\lor γ\lor σ}
a\lor γ\lor b\lor σ
\item Сокращение
\uline{a\lor b\lor b\lor γ}
a\lor b\lor γ
\end{enumerate}
\item Сильные
\begin{enumerate}
\item Ослабление
\uline{γ}
a\lor γ
\item Де-Морган
\uline{(\lnot a)\lor γ$_{\text{(\lnot b)}}$\lor γ}
\lnot (a\lor b)\lor γ
\item Отрицание
\uline{a\lor γ\_\_}
(\lnot \lnot a)\lor γ
\item Квантификация
\uline{\lnot a(t)\lor γ}
\lnot \forall x.a(x)\lor γ
\item Бесконечная индукция
\uline{a(0\textasciitilde{})\lor γ$_{\text{a}}$(1\textasciitilde{})\lor γ} \dots  \uline{a(r\textasciitilde{})\lor γ}\dots
\forall xa(x)\lor γ
\end{enumerate}
\item Сечение (для облегчения жизни)
\uline{γ\lor a$_{\text{\lnot a\lor δ}}$}
γ\lor δ
\end{enumerate}
\item Порядки
Каждой формуле в дереве соответствует порядок, причем
посылке и заключению (выше и ниже \uline{\_}) слабого правила
вывода соответствует один порядок, а порядковое число,
отнесенное заключению сильного правила или сечения, больше
порядковых чисел, отнесенных соотвтетствующим посылкам.
Порядковые числа - это ординалы, они могут быть достижимыми,
но не конечными - пусть формула какая-нибудь околорекурсивная
типа Ф(x): Ф(0) = A \to A, Ф(1) = A \to A \to A, Ф(2) = A \to A \to A \to A.
Тогда пусть мы хотим доказать \forall x.Ф(x) - по бесконечной
индукции порядок термов будет увеличиваться, а порядок
\forall x.Ф(x) будет w. Именно из-за этого факта мы используем
в доказательстве теоремы об устранении сечений трансфинитную
индукцию по порядку - ведь обычной индукции мало для порядков
больших w.
\item Степень
Степень сечения - количество связок в \lnot a.
Степень доказательства - наибольшая степень сечения в дереве.
Степень всегда конечна - любая формула в ФА содержит конечное
число связок, а при трансляции нет возможности увеличить
их количество. Тогда трансфинитная индукция по термам, в
которых в сечении количество связок растет до бесконечности,
невозможно.

\item Нитью называется последовательность формул от начальной до
конечной. Все нити в доказательстве конечны, поскольку если в начальной
формуле стоит ординал, последовательность в нити не возрастает, а эти
числа убывают с применением строгого правила или сечения. Мы знаем,
что строго убывающая бесконечная последовательность ординалов не существует.
Добавим правило, что последовательность применения слабых правил подряд
была всегда конечна
\item Теорема в S∞ - выражение, которое может стоять в заключительной формуле
вывода
\end{enumerate}
\subsection{Теоремы об эквивалентности ФА и S∞}
\label{sec-18-3}
\subsubsection{Лемма 1: В S∞ выводимо A\lor \lnot A}
\label{sec-18-3-1}
А либо корректна, либо некорректна, тогда
\uline{A}
\lnot A\lor A     ослабление

\uline{\lnot A}
\uline{A\lor \lnot A}   ослабление
\uline{\lnot A\lor A}   перестановка
\subsubsection{Лемма 2: В S∞ выводимо s\ne t\lor \lnot A(s)\lor A(t)}
\label{sec-18-3-2}
Если выводимо A(s), s = t, то выводимо A(t) (все вхождения меняем)
если s=t, то выводимо \lnot A(t)\lor A(t), потом сделаем ослабление
если s\ne t, то она аксиома (некорректная) и ослабим.
\subsubsection{Лемма 3: всякая выводимая в S замкнутая формула А является теоремой S∞}
\label{sec-18-3-3}
Докажем, что если что-то доказуемо в ФА, то его эквивалент
доказуем и в S∞.
$\vdash ₚₐA => \vdash ₛA'$
Схема док-ва
Рассмотрим доказательство в ФА, оно состоит из
\beta_1'\dots \beta_n', оттранслируем каждое в \beta_i \in S∞ (по полноте \{\lnot , \lor \} это возможно)
Тогда можно сделать дерево, в котором начальные формулы - аксиомы S,
а правила вывода - MP, GEN.

Рассмотрим формулу A = \beta_i
\begin{enumerate}
\item B \to C \to B, те \lnot B \lor (\lnot  C \lor B)
Из замкнутости A следует замкнутость B
\lnot B \lor B выводима по Л1, тогда ослабим с \lnot C, переставим.
\item (B \to C) \to (B \to C \to D) \to B \to D
\lnot (\lnot B \lor C) \lor (\lnot (\lnot B \lor (\lnot C \lor D)) \lor (\lnot B \lor D))
По Л1 выводимо \lnot (\lnot B \lor C) \lor (\lnot B \lor C),
(\lnot B \lor \lnot C \lor D) \lor \lnot (\lnot B \lor \lnot C \lor D)
Тогда можно по перестановке, сечению (с С) и сокращению
доказать (B \to C \to D) \to (B \to C) \to (B \to D)
что одно и то же, см дедукцию в предикатах
\item (B \to C) \to (B \to \lnot C) \to \lnot B
\lnot (\lnot B \lor C) \lor \lnot (\lnot B \lor \lnot C) \lor \lnot B
\begin{enumerate}
\item \lnot B \lor B                        л1
\item \lnot \lnot \lnot B \lor B                      отрицание
\item \lnot (\lnot \lnot B \lor C) \lor \lnot \lnot \lnot B \lor B         ослабление
\item \lnot \lnot \lnot B \lor \lnot (\lnot \lnot B \lor C) \lor B         перестановка
\item \lnot \lnot \lnot B \lor B \lor \lnot \lnot C                аналогично + еще перестановка
\item \lnot C \lor \lnot \lnot C                      лемма
\item \lnot C \lor B \lor \lnot \lnot C                  ослабление + перестановка
\item \lnot (\lnot \lnot B \lor C) \lor B \lor \lnot \lnot C          де-морган от 6 и 8
\item \lnot \lnot C \lor \lnot (\lnot \lnot B \lor C) \lor B          перестановка 8
\item \lnot (\lnot \lnot B \lor \lnot C) \lor \lnot (\lnot \lnot B \lor C) \lor B де-морган от 4 и 9
\end{enumerate}
Ну вот мы доказали что-то очень похожее на то, что нужно было.
Там контрпозиция, шмяк шмяк, готово.
\item Видимо, примерно все формулы так доказываются.
\item \forall x.B(x) \to B(t)
\lnot \forall x.B(x) \lor B(t)
По л1 \lnot B(t) \lor B(t), потом квантификация по 1 элем.
\item B(t) \to \exists x.B(x)
B(t) \to \lnot \forall x.B(x) (что заметно отличается от \forall x.\lnot B(x))
\lnot B(t) \lor \lnot \forall x.B(x)
Не, я не знаю. Но точно можно! По бесконечной индукции мож как-то.
Или там сечение хитрое.
\end{enumerate}

А еще есть аксиомы ФА
\begin{enumerate}
\item a = b \to a' = b'
\lnot (a = b) \lor a' = b'
\begin{itemize}
\item если a = b, то a' = b', это аксиома S∞,
тогда по ослаблению добавим \lnot (a = b)
\item если a \ne  b, то она же и аксиома
\end{itemize}
\item a = b \to a = c \to b = c
a \ne  b \lor a \ne  c \lor b = c
a \ne  b \lor \lnot (x = c) @ b \lor (x = c) @ c     по лемме 2
\item a' = b' \to a = b
Аналогично 1
\item \lnot (a' = 0)
Аксиома, поскольку a' всегда имеет с 0 разные значения
\item a + b' = (a + b)'
TODO
\item a + 0 = a
Аксиома, поскольку это вседа равенство
\item a * 0 = 0
аналогично 6
\item a * b' = a * b + a
TODO
\item \phi[x:=0] \& \forall x.(\phi \to \phi[x:=x']) \to \phi
\lnot B(0) \lor \lnot \forall x(\lnot B(x)\lor B(x')) \lor B(0)         лемма 1 и перестановка
\lnot B(0) \lor \lnot (\lnot B(0) \lor B(1)) \lor \dots            можно показать по индукции
\dots  \lor \lnot (\lnot B(k) \lor B(k')) \lor B(k')   (ослабление, перестановка, де-морган)
\lnot B(0) \lor \lnot (\forall x(\lnot B(x) \lor B(x'))) \lor B(k')    k раз квантификация, перестановки, сокращ.
Применим бесконечную индукцию относительно первого
и третьего терма и получим что надо.
\end{enumerate}

Окей, с аксиомами разобрались.
И еще есть два правила вывода
\begin{enumerate}
\item MP
B            условие
\lnot B \lor A       условие
A            сечение
\item GEN
B(x)         условие
Продвигаясь от этой формулы вверх можно поменять все
x на k, тогда верно
$$\vdash B(k)$$
На основании принципа бесконечной индукции доказываем \forall xB(x)
\item A \to B(t) => A \to \forall x.B(x)
\lnot A \lor B
Заменим все вхождения перменной в доказательстве в
ФА формулы \lnot A \lor B на 0, 1, 2\dots ,
тогда по бесконечной индукции:
$\lnot A \lor B(0), \lnot A \lor B(1), \dots  \vdash \lnot A \lor \forall x.B$   (только B(0) \lor \lnot A везде)
\item A(t) \to B => \exists x.A(x) \to B
$\lnot A \lor B \vdash \lnot \lnot \forall x.(\lnot A(x)) \to B$
Заменим все вхождения свободной перменной t в ФА на конкретные.
Получим счетное мн-во д-в \lnot A(0) \lor B, \lnot A(1) \lor B, \dots
по беск. индукции
\forall x.\lnot A(x) \lor B
\lnot \lnot \forall x.\lnot A(x) \lor B -  навесили двойное отрицание
\end{enumerate}
\subsubsection{Следствие: непротиворечивость S∞ влечет непротиворечивость S}
\label{sec-18-3-4}
Пусть в S доказуемо \lnot (0=0), тогда оно доказуемо и в
S∞, тогда
\uline{A\lor 0=0$_{\text{\lnot }}$(0=0)\lor A}          Аргументы получаются по ослаблению
\uline{A\lor A}
A
\subsection{Теорема Генцена об устранении сечений}
\label{sec-18-4}
\subsubsection{Лемма: сильные правила 2, 3, 5 обратимы}
\label{sec-18-4-1}
Правила обратимы и их д-во имеет порядок и степень не больше,
чем первоначальное
Нам дают доказательство формулы, мы строим новое дерево,
в котором из результата следуют посылки.
\begin{enumerate}
\item Правило Де-Моргана
\lnot (B \lor E) \lor D
Рассмотрим вывод формулы. Проследим вхождения
подформул \lnot (B \lor E)  формулы, которые соответствуют
вхождению в конечную формулу (то есть не те, которые
исчезают в сечении). Мы пройдем через всякий случай
применения слабого правила и сильные, когда \lnot (B \lor E)
является боковой формулой. Остановка произойдет
либо в ослаблении $( F \vdash \lnot (B \lor E) \lor F )$ либо в Де-Моргане
$( \lnot B \lor F, \lnot E \lor F \vdash \lnot (B \lor E) \lor F )$. Совокупность таких
вхождений формулы назовем ее \textbf{историей}.
Если все вхождения формулы \lnot (B \lor E) в ее истории заменить
на \lnot B, то в результате получится вывод \lnot B \lor D. Аналогично
можем вывести \lnot E \lor D.
\item Правило отрицания
\lnot \lnot B \lor D
Заменим все вхождения \lnot \lnot B в ее истории на B, получим
вывод B \lor D
\item Бесконечная индукция
\forall xB(x) \lor D
Заменим все вхождения \forall xB(x) в ее истории (1 элемент) на
B(k) (причем если мы уткнулись в бесконечную индукцию,
выберем только ту ветку, которая соответствует B(k))
Тогда для любого k получим вывод B(k)
\end{enumerate}
\subsubsection{Теорема: устранение сечения}
\label{sec-18-4-2}
Если для A в S∞ существует вывод (m, a), то существует
вывод в S∞ (n, 2ᵃ), где n < m
Докажем через трансфинитную индукцию по порядку a вывода A
\begin{itemize}
\item База: порядок вывода 0
Вывод не содержит сечений и его степень 0
\item Переход: пусть теорема верна для выводов порядков меньших a
Будем продвигаться вверх по выводу, пока не встретим первое
применение сильного правила или сечения.
\begin{enumerate}
\item Сильное правило
Пусть его посылки занумерованы порядковыми числами aₗ
Согласно индуктивному предположению, для этих посылок
существует дерево вывода F со степенью < m и порядком 2$^{\text{(a_i)}}$.
Заменим таким деревом то поддерево данного дерева вывода,
заключительной формулой которого служит рассматриваемое
вхождение F. Сделав так со всеми посылками мы получим новое
дерево для A, отнесем ему порядковое число 2ᵃ > 2$^{\text{(a_i)}}$
(пояснение от меня, потому что я чет долго доганял - каждую
посылку заменяем по индукции на ее модное новое дерево
от нее самой же).
\item Сечение
Значит имеем что-то на уровне
\uline{C\lor B\_$_{\text{\lnot B\lor D}}$}
C\lor D
Согласно индуктивному предположению для C\lor B и \lnot B\lor D существуют
выводы степеней меньших m и порядков 2$^{\text{a_1}}$, 2$^{\text{a₂}}$. Рассмотрим
разные случаи строения B:
\begin{enumerate}
\item B - это элементарная формула. Одна из формул B и \lnot B есть
аксиома. Пусть K - та, которая не аксиома.
По предположениию поддерево основного дерева вывода с K
может быть заменено другим со степенью n и порядка 2$^{\text{(a}_{\text{i}}\text{)}}$
(i = 1 или 2, смотря в какой посылке K).
В этом новом дереве расмотрим историю K, начальные формулы
в которой могут возникнуть только по ослаблению (а исчезать
по сечению). Поэтому удаление всех вхождений K из истории
приводит к построению дерева вывода для D или C порядка
2$^{\text{a_i}}$. Отсюда с помощью ослабления получаем дерево вывода
для C \lor D порядка 2$^{\text{a}}$. Степень меньше m (одно сечение убрали).
\item B - это \lnot E, тогда посылки выглядят как
C\lor \lnot E, \lnot \lnot E\lor D
Существует дерево вывода для \lnot \lnot E\lor D степени < m и порядка 2$^{\text{a₂}}$
В силу леммы об обратимости можно построить дерево вывода
E\lor D порядка 2$^{\text{a₂}}$ степени < m
Кроме того существует дерево вывода степени < m и порядка 2$^{\text{a_1}}$
для левой посылки, тогда построим из них новое дерево
вывода
\uline{E\lor D}   \uline{C\lor \lnot E}
\uline{D\lor E\_$_{\text{\lnot E\lor C}}$}
\uline{D\lor C}
C\lor D
Степень выделенного здесь сечения на единицу меньше общего
числа связок и кванторов в \lnot E, которое само по себе \le m.
Формуле C\lor D можно присвоить порядковое число C \lor D по свойству
посылок новог сечения.
\item B - это E\lor F, посылки:
C\lor E\lor F, \lnot (E\lor F)\lor D
Существует дерево вывода для правой посылки < m и порядка 2$^{\text{a₂}}$,
по лемме об обратимости существуют выводы степеней < m и порядка
2$^{\text{a₂}}$ для \lnot E\lor D и \lnot F\lor D (по Де-Моргану). По предположению индукции
есть еще дерево вывода для левой посылки.
Из последних трех построим такое:
\uline{C\lor E\lor F\_$_{\text{\lnot F\lor D}}$}
   \uline{C\lor E\lor D}
   \uline{C\lor D\lor E\_\_\_$_{\text{\lnot E\lor D}}$}
       \uline{C\lor D\lor D}
C\lor D
Степень сечения уменьшили (в каждом на 1), в формуле
C\lor E\lor D можно дать порядок $2^{\max(a_1, a₂)}$+1, а остальным - $2^a$
\item B - это \forall xE, посылки:
C\lor \forall xE  (\lnot \forall xE)\lor D
По индуктивному предположению для левой посылки можно построить
дерево вывода (< m,2$^{\text{a_1}}$). В силу указания в начале 2 леммы о
эквивалентности S∞ и ФА (если $\vdash A(t), t=s, то \vdash A(s)$) и леммы
об обратимости для любого постоянного терма z существует вывод
C\lor E(z).
Можем и правую посылку заменить (< m, 2$^{\text{a₂}}$) по инд.предположению
Тогда в правой посылке история \lnot (\forall xE) может начинаться либо
с ослабления либо с квантификации.
Заменим на C все такие начала - если это ослабление, то просто
подменим вместо \lnot (\forall xE) новое C, если квантификация, то
\uline{\lnot E(t)\lor F\_\_}         \uline{C\lor E(t)\_$_{\text{\lnot E}}$(t)\lor F}   //первое мы взяли из левой посылки
\lnot (\forall xE(x))\lor F   =>          C\lor F
В результате мы получили дерево вывода C\lor D
Степень меньше m, поскольку связку мы одну убрали
\textbf{тут еще ОЧЕНЬ АДОВЫЕ оценки порядка C\lor D, но он 2$^{\text{a}}$}
\end{enumerate}
\end{enumerate}
\end{itemize}
\subsubsection{Следствие: устранение всех сечений}
\label{sec-18-4-3}
Воспользуемся леммой об устранении сечения, пока степень вывода
не станет равна нулю. Тогда порядок будет что-то на уровне степенной
башни из двоек, в вершине которой первоначальная степень, а количество
двоек -- количество применения леммы.
\subsubsection{Следствие: S∞ непротивроечива}
\label{sec-18-4-4}
Если S∞ противоречива, в ней докажется $0\ne 0\lor 0\ne 0\lor \dotsb \lor 0\ne 0$.
В ее истории появление 0\ne 0 может быть только из-за ослабления,
но тогда посылкой ослабления тоже будет 0\ne 0. Если в д-ве есть
сечения, мы можем перестроить его, устранив все сечения, а затем провести
рассуждения вновь и увидеть, что такая формула недоказуема, а значит
S∞ непротиворечива, значит и ФА тоже непротиворечива.
\section{Ключевые фигуры}
\label{sec-19}
\begin{itemize}
\item Станислав Яськовски - 1906
Aлгебра Яськовского, один из первых исследователей ИИВ
\item Герхард Генцен - 1909
Теорема об устранении сечения (1935)
\item Курт Гёдель - 1906
Теоремы о непротиворечивости - 1930
Чуть ли не все остальное
\item Сол Крипке - 1940
Семантика Крипке - 1960-1970
\item Дэвид Гильберт - 1862, Пауль Бернайс - 1888
Основания математики - 1934, 39
\item Их достаточно много, а времени мало.
\end{itemize}


\end{document}
