\section{Ticket 13: ТМ}
\label{sec-15}
\subsection{Теория множеств}
\label{sec-15-1}
   Значит это такая теория первого порядка.
   В сигнатуре модели есть один пред.символ - \in
   Добавляем связку a \leftrightarrow b = (a \to b) \& (b \to a)
   σ \in Θ => \forall x(x \in σ \to x \in Θ)
o   σ = Θ => σ \in Θ \& Θ \in σ
   ∅ : \forall x(\lnot x \in ∅)
   x ∩ y = z, тогда \forall t(t \in z \leftrightarrow t \in x \& t \in y)
   Dj(x) \forall a\forall b(a \in x \& b \in x \& a \ne  b \to a ∩ b = ∅)
   X(a) - мн-во всех x пересекающихся ровно в одном эл-те с каждым из а
   и содержащих элементы из ∪a.
   X(\{\{1, 2\}, \{2', 3\}\}) = \{\{2, 3\}, \{1, 2'\}\}
\subsubsection{Если существует мн-во, то существует пустое мн-во}
\label{sec-15-1-1}
Аксиома выделения:
\forall x\exists b\forall y(y \in b \leftrightarrow (y \in x \& \phi(y)))
Возьмем наше существующее мн-во x
\exists b\forall y(y \in b \leftrightarrow (y \in x \& \phi(y)))
Пусть \phi(y) = \bot
тогда подставим ∅ вместо b:
\forall y(y \in ∅ \leftrightarrow (y \in x \& \bot))
Это выполняется вроде.
\subsubsection{Если x, то найдется \{x\}}
\label{sec-15-1-2}
\forall x\exists \{x\}\forall y(y \in \{x\} \to y = x)
\begin{enumerate}
\item Пусть x \ne  ∅
\{x\} = \{y | y \in \{x, y\} \& y \ne  ∅\}
по аксиоме объединения \exists p\forall y(y \in p \leftrightarrow \exists s(y \in s \& s \in x))
\forall y(y \in \{x, ∅\} \leftrightarrow \exists s(y \in s \& s \in x))

или по аксиоме пары
\exists p(x \in p \& ∅ \in p \& \forall z(z \in p \to (x = z \lor y = z)))
x \in \{x, ∅\} \& ∅ \in \{x, ∅\} \& \forall z(z \in \{x, ∅\} \to \dots \}

ДГ руками помахал тут, ну и я помахаю по причине
отсутствия времени доказывать

А, нет, вот, кажется:
По аксиоме степени \forall x\exists \{x, ∅\}\forall y(y \in \{x, ∅\} \leftrightarrow y \in x)
\forall x\exists \{x, ∅\}\forall y((y \in \{x, ∅\} \to y \in x)\&(y \in x \to y \in \{x, ∅\}))
\lnot y \in ∅, значит (y \in x \to y \in x) = \top
\forall x\exists \{x, ∅\}\forall y((y \in \{x, ∅\} \to y \in x)\&\top)
\forall x\exists \{x, ∅\}\forall y(y \in \{x, ∅\} \to y = x)   более слабое условие
\item x = ∅
P(∅) = \{∅\}
\end{enumerate}
\subsubsection{\exists !x(\forall y.\lnot (y \in x))}
\label{sec-15-1-3}
\exists x(\forall y.\lnot (y \in x)) \& \forall a\forall b((\forall y.\lnot (y \in a)) \& (\forall y.\lnot (y \in b)) \to a = b)
Первое по определению пустого множества и аксиоме выделения с \bot
\forall y.(\lnot y \in \{\}) \& \forall y.(\lnot y \in \{\}) \to \forall p((p\in x \to p\in y) \& (p\in y \to p\in x))
Второе как-то через ∅_1 \in ∅₂ и обратное включение
На основании того, что мы подставляем наши пустые множества, импликация
вырождается в \top \to \top
\subsubsection{x ∩ y существует}
\label{sec-15-1-4}
по теореме выделения
\forall x\exists b\forall y(y \in b \leftrightarrow (y \in x \& \phi(y)))
\forall y(y \in x ∩ y \leftrightarrow (y \in x \& t \in y))
\subsection{Аксиоматика ZFC}
\label{sec-15-2}
\subsubsection{Аксиома равенства}
\label{sec-15-2-1}
\forall x\forall y\forall z((x = y \& y \in z) \to x \in z)
Eсли два множества равны, то любой элемент лежащий в первом,
лежит и во втором
\subsubsection{Аксиома пары}
\label{sec-15-2-2}
\forall x\forall y(\lnot (x=y) \to \exists p(x \in p \& y \in p \& \forall z(z \in p \to (x = z \lor y = z))))
x \ne  y, тогда сущ. \{x, y\}
\subsubsection{Аксиома объединений}
\label{sec-15-2-3}
\forall x(\exists y(y\in x) \to \exists p\forall y(y \in p \leftrightarrow \exists s(y \in s \& s \in x)))
Eсли x не пусто, то из любого семейства множеств можно
образовать „кучу-малу“, то есть такое множество p,
каждый элемент y которого принадлежит по меньшей мере
одному множеству s данного семейства s x
\subsubsection{Аксиома степени}
\label{sec-15-2-4}
\forall x\exists p\forall y(y \in p \leftrightarrow y \in x)
P(x) - множество степени x (не путать с 2ˣ - булеаном)
Это типа мы взяли наш x, и из его элементов объединением и
пересечением например понаобразовывали кучу множеств, а потом
положили их в p.
\subsubsection{Схема аксиом выделения}
\label{sec-15-2-5}
\forall x\exists b\forall y(y \in b \leftrightarrow (y \in x \& \phi(y)))
Для нашего множества x мы можем подобрать множество побольше,
на котором для всех элементов, являющихся подмножеством x
выполняется предикат.
\subsubsection{Аксиома выбора (не входит в ZF по дефолту)}
\label{sec-15-2-6}
Если a = Dj(x) и a \ne  0, то x \in a \ne  0
\subsubsection{Аксиома бесконечности}
\label{sec-15-2-7}
\exists N(∅ \in N \& \forall x(x \in N \to x ∪ \{x\} \in N))
\subsubsection{Аксиома фундирования}
\label{sec-15-2-8}
\forall x(x = ∅ \lor \exists y(y \in x \& y ∩ x = ∅))
\forall x(x \ne  ∅ \to \exists y(y \in x \& y ∩ x = ∅))
Равноценные формулы.

Я бы сказал, что это звучит как-то типа
"не существует бесконечно вложенных множеств"
\subsubsection{Схема аксиом подстановки}
\label{sec-15-2-9}
\forall x\exists !y.\phi(x,y) \to \forall a\exists b\forall c(c \in b \leftrightarrow (\exists d.(d \in a \& \phi(d, c))))
Пусть формула \phi такова, что для при любом x найдется единственный y
такой, чтобы она была истинна на x, y, тогда для любого a
найдется множество b, каждому элементу которого c можно сопоставить
подмножество a и наша функция будет верна на нем и на c
Типа для хороших функций мы можем найти множество с отображением из
его элементов в подмножество нашего по предикату.

